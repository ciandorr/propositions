% \iffalse meta-comment
%
%% File: propositions.dtx
%%
%% Copyright (C) 2026 Cian Dorr
%%
%% This work may be distributed and/or modified under the
%% conditions of the LaTeX Project Public License, either version 1.3c
%% of this license or (at your option) any later version.
%% The latest version of this license is in
%%
%%    https://www.latex-project.org/lppl.txt
%%
%% This work consists of the files propositions.dtx and propositions.ins
%% and the derived file propositions.sty.
%%
% \fi
%
% \iffalse
%<*driver>
\documentclass[leqno]{ltxdoc}
\usepackage[T1]{fontenc}
\usepackage{amsmath}
\usepackage[most]{tcolorbox}
\tcbuselibrary{documentation}
\usepackage[equations, tightspacing, leftmargin=1em, labelindent=0em]{propositions}
\usepackage{cleveref}
\usepackage{fancyvrb}
\VerbatimFootnotes

\hypersetup{colorlinks,linkcolor=blue!60!black,urlcolor=blue!60!black}

%% Side-by-side example environment (fancyvrb-based, works inside \DocInput).
\newenvironment{example}
  {\VerbatimEnvironment
   \begin{VerbatimOut}{\jobname-ex.tex}}
  {\end{VerbatimOut}%
   \par\medskip\noindent
   \begin{minipage}[t]{0.52\linewidth}%
     \small\VerbatimInput[gobble=2,frame=leftline,
       framerule=0.4pt,framesep=1.5em]{\jobname-ex.tex}%
   \end{minipage}\hfill
   \begin{minipage}[t]{0.44\linewidth}%
     \input{\jobname-ex.tex}%
   \end{minipage}%
   \par\medskip}

\tcbset{
  docexample/.style={
    colframe=blue!50!black,
    colback=blue!3!white,
    before skip=\medskipamount,
    after skip=\medskipamount,
    fontlower=\small,
  },
}


\EnableCrossrefs
\CodelineIndex
\OnlyDescription
\begin{document}
  \DocInput{\jobname.dtx}
\end{document}
%</driver>
% \fi
%
% \GetFileInfo{propositions.sty}
%
% \title{The \textsf{propositions} package\thanks{This document describes
%   version~\fileversion, dated~\filedate.}}
% \author{Cian Dorr (with help from Claude Code)\\ \texttt{ciandorr@gmail.com}}
% \date{\filedate}
%
% \maketitle
%
% \begin{abstract}
% \noindent The \textsf{propositions} package provides a key-value driven system
% for labelling propositions, theses, and premises in academic papers.
% Items may be given names like `(P)' or
% `Physicalism', or auto-numbered using different counters; all carry robust
% cross-references with configurable formatting.
% The package integrates with \textsf{amsmath}, \textsf{hyperref},
% and \textsf{cleveref}.
% \end{abstract}
%
% \tableofcontents
%
% \section{Introduction}
%
% In some academic disciplines (such as philosophy), it is common to have displayed propositions (examples, theses, premises,\ldots) with various kinds of labels.  A thesis might be referred to as `(P)' or `Physicalism'; the premises of an argument might be numbered as `P1', `P2', `P3', \ldots; or examples might be numbered consecutively over the course of a whole article.
%   Standard \LaTeX{} environments like |enumerate| can
% handle some of these cases, but cross-referencing is awkward:
% |\ref| produces a bare number or letter, and the author must manually
% add parentheses or other formatting at every point of reference.  The standard |description| environment does not allow cross-referencing at all.
%
% The \textsf{propositions} package solves this by attaching formatting
% information to each label.  A short item like |\pitem[P]| is displayed
% as ``(P)'' and |\ref| automatically produces ``(P)'' as well---complete
% with parentheses, hyperlinks, and \textsf{cleveref} support.  The full
% key-value interface supports named items, numbered items, custom
% counters, glosses, shorthands, and per-item format overrides.
%
% The |\ptag| command (which requires \textsf{amsmath}) extends this to
% displayed math environments: an equation can be tagged with a
% proposition label instead of (or using) its equation number.
% 
% \section{History}
% I wrote the ancestor to this package in the 90s
% while finishing my Ph.D. thesis, but never documented it 
% or shared it with the world.  This new 
% version is a thorough re-implementation in \LaTeX3,
% written in 2026 with extensive help from Claude Code.  I 
% hope others will find it as useful as I have. 
% 
% \section{Basic usage}
% Load the package with |\usepackage{propositions}| or |\usepackage|\oarg{options}|{propositions}| (see \autoref{sect:global} below for valid package options).
%
% The |prop| environment generates a list of propositions, each introduced by a |\pitem|.  |\pitem| with an optional argument gives a |description|-like label:
% \begin{example}
% \begin{prop}
%   \pitem[Physicalism] Everything is
%   physical. \label{phys}
%   \pitem[Idealism] Everything is
%   mental. \label{ideal}
% \end{prop}
% \end{example}
% Unlike the standard |description| environment, one can refer back to these propositions using the standard |\ref| command:
% \begin{example}
% \ref{phys} is more plausible
% than \ref{ideal}.
% \end{example}
% With no optional argument, |\pitem| will by default generate numbered items similar to |enumerate|, but with numbering that persists across the document:
% \begin{example}
% \begin{prop}
%   \pitem Every atom is physical.
%   \label{atoms}
% \end{prop}
% \ref{phys} follows from the
% conjunction of \ref{atoms} and
% \begin{prop}
%   \pitem Everything is an atom.
%   \label{atomism}
% \end{prop}
% \end{example}
% As with |enumerate|, the
% counter and formatting depend on the
% nesting level:
% \begin{example}
% \begin{prop}
%   \pitem \label{dual}
%   \begin{prop}
%     \pitem Some things are physical.
%     \label{some}
%     \pitem Some things are not
%     physical. \label{notall}
%   \end{prop}
% \end{prop}
% Without \ref{some}, \ref{dual}
% would be consistent with \ref{ideal}.
% \end{example}
%
% \section{Advanced usage}
% The format of the proposition labels, and of
% subsequent references, are both configurable using a key=value syntax (see \autoref{sect:keys} for the possible keys):
% \begin{example}
% \begin{prop}
%   \pitem[No Overlap,
%     align=flush,
%     display format=\textbf{#1},
%     ref format=\textit{#1}]
%     Nothing mental is
%     physical. \label{incomp}
% \end{prop}
% Is \ref{dual} consistent
% with \ref{phys}?
% \end{example}
% Preset \emph{item types} can be declared and used instead of configuring the |\pitem|s one at a time:
% \begin{example}
% \begin{prop}
%   \pitem[Nihilism,
%     type=long]
%     There is nothing.
%     \label{nihilism}
% \end{prop}
% Does \ref{nihilism} imply
% \ref{phys}, \ref{dual}, or
% both?  Discuss.
% \end{example}
% Shortcut commands---e.g., |\litem[|\meta{keys}|]| for |\pitem[type=long, |\meta{keys}|]|---can also be defined.  The package loads with a range of predefined types.
%
% When the package is loaded with |\usepackage[equations]{propositions}|, first-level |\pitem|s use the same counter
% as equations.  (This looks better with the |leqno|
% option to |\documentclass|.)
% \begin{example}
% \begin{equation}
%   \exists x (\text{Mental}(x)
%   \wedge \text{Physical}(x))
% \end{equation}
% \begin{prop}
%   \pitem
%   There is overlap between
%   the mental and the physical.
% \end{prop}
% \end{example}
% The |\ptag| command (requires \textsf{amsmath})
% is an analogue of |\pitem| that works inside
% displayed math environments.
% \begin{example}
% \begin{equation}
%   \ptag[Monism] \label{mon}
%   \exists x \forall y(y = x)
% \end{equation}
% Is \ref{mon} compatible with
% \ref{dual}?
% \end{example}
%
% \section{The \texttt{prop} environment and \cs{pitem}}
% \label{sect:keys}
%
% \begin{docEnvironment}{prop}{\oarg{keys}}
%   Creates a displayed list of propositions.  It is a standard \LaTeX{} list, so by default its formatting will depend on the standard length parameters like \cs{itemsep} and \cs{topsep}, although these can be overridden by setting package keys.
%
%   Within \texttt{prop}, \cs{pitem} (see below) creates labelled items.  The ordinary \cs{item} command is still available for unlabelled items.
%
%   The optional \meta{keys} argument accepts the same keys as \refCom{propoptions} (\autoref{sect:global}), with effects local to this environment.
% \end{docEnvironment}
%
% \begin{docEnvironment}{inlineprop}{\oarg{keys}}
%   Like \texttt{prop}, but does not create a list.  Allows \cs{pitem} to be used outside list environments, e.g.\ for generating numbers at the beginning of paragraphs.  Steps the \texttt{prop} counter and increments the nesting level.  Accepts the same optional \meta{keys} as \texttt{prop}.
% \end{docEnvironment}
%
% \begin{docCommand}{pitem}{\oarg{keys}}
%   Inside the \texttt{prop} and \texttt{inlineprop} environments, introduces a labelled proposition.  The optional argument is a comma-separated list of \meta{key}=\meta{value} pairs.
%
%   When used without an optional argument (or without setting \texttt{name}, \texttt{counter}, or \texttt{type}), it behaves as \cs{pitem}\texttt{[type=}\meta{type}\texttt{]}, where \meta{type} depends on the nesting depth.  The defaults are \texttt{numbered}, \texttt{leveltwo}, \texttt{levelthree}, \texttt{levelfour}, \texttt{levelfive}; these can be changed with the \texttt{level}~\meta{n} keys (\autoref{sect:global}).
% \end{docCommand}
%
% The following keys can be used in the optional argument of |\pitem|:
%
% \begin{docKey}{name}{=\meta{text}}{no default}
%   The proposition's name.  A bare string (without \texttt{=}) in the key list is equivalent to \texttt{name=}\meta{text}.
% \end{docKey}
%
% \begin{docKey}{type}{=\meta{type}}{no default}
%   An item type, equivalent to a preset collection of keys.  Types can be declared using \refCom{SetItemType} or \refCom{DeclareNumberedType}, and several come predefined (\autoref{sect:types}).
% \end{docKey}
%
% \begin{docKey}{counter}{=\meta{name}}{no default}
%   Counter to use.  The counter is automatically stepped, and the item's \texttt{name} is set to \cs{the}\meta{name}, though this can be overridden by explicitly setting \texttt{name}.
% \end{docKey}
%
% \begin{docKey}{align}{=\meta{type}}{no default}
%   How the label should be positioned.  Possible values: \texttt{default} (offset controlled by \cs{labelwidth} and \cs{labelsep}), \texttt{right} (right-aligned, like \texttt{enumerate}), \texttt{left} (aligned with left margin of surrounding text), \texttt{flush} (aligned with left margin of item text), \texttt{nextline} (label on its own line), \texttt{left-nextline}, and \texttt{flush-nextline}.
%   Has no effect inside \texttt{inlineprop} or \cs{ptag}.
% \end{docKey}
%
% \begin{docKey}{format}{=\meta{template}}{no default}
%   Formatting applied to the \texttt{name}: use \texttt{\#1} for the argument, e.g.\ \texttt{format=\string\textbf\{(\#1)\}}.  Shorthand for setting both \texttt{display format} and \texttt{ref format}.
% \end{docKey}
%
% \begin{docKey}{display format}{=\meta{template}}{no default}
%   Format for displaying the name in the proposition's label.  Does not affect cross-references.
% \end{docKey}
%
% \begin{docKey}{ref format}{=\meta{template}}{no default}
%   Format for subsequent cross-references to this proposition.  Does not affect the display.
% \end{docKey}
%
% \begin{docKey}{shorthand}{=\meta{text}}{no default}
%   An abbreviation displayed after the name.  If present, the shorthand becomes the reference text: \cs{ref} produces the shorthand (formatted with \texttt{ref format}) rather than the full name.
% \end{docKey}
%
% \begin{docKey}{shorthand format}{=\meta{template}}{initially \texttt{\string~[\#1]}}
%   Format for displaying the shorthand in the label.
% \end{docKey}
%
% \begin{docKey}{gloss}{=\meta{text}}{no default}
%   A parenthetical gloss displayed after the name.  Does not affect cross-references.
% \end{docKey}
%
% \begin{docKey}{gloss format}{=\meta{template}}{initially \texttt{\string~(\#1)}}
%   Format for displaying the gloss in the label.
% \end{docKey}
%
% \begin{docKey}{ref}{=\meta{text}}{no default}
%   Explicitly set the reference text, overriding what would be derived from \texttt{name}, \texttt{counter}, or \texttt{shorthand}.
% \end{docKey}
%
% \begin{docKey}{label}{=\meta{label}}{no default}
%   Equivalent to a trailing \cs{label}\marg{label}.
% \end{docKey}
%
% \begin{docKey}{crefname}{=\meta{type}}{no default}
%   When \textsf{cleveref} is loaded, assigns an arbitrary reference type to this proposition.  For example, \texttt{crefname=lemma} causes \cs{cref} to use the names defined by \cs{crefname}\texttt{\{lemma\}\{...\}\{...\}} instead of the default \texttt{proposition} type.  The \meta{type} must be known to \textsf{cleveref}; new types can be declared with \cs{crefname}.
% \end{docKey}
%
% \begin{docCommand}{ptag}{\oarg{keys}}
%   Available only when \textsf{amsmath} is loaded.  Works inside displayed math environments like \texttt{equation} and \texttt{align}.  Accepts the same keys as \refCom{pitem}, except that \texttt{align} has no effect (positioning is controlled by the tag placement system).
% \end{docCommand}
%
% \section{Item types}\label{sect:types}
%
% \begin{docCommand}{SetItemType}{\marg{name}\marg{keys}}
%   Defines or modifies an item type for use with the \texttt{type} key.  All \refCom{pitem} keys are accepted, plus the following:
% \end{docCommand}
%
% \begin{docKey}{macro}{=\meta{command}}{no default}
%   A new user macro, equivalent to \cs{pitem}\texttt{[type=}\meta{name}\texttt{]}.  Any further keys given to the macro are passed to \cs{pitem}.
% \end{docKey}
%
% If the type \meta{name} already exists, |\SetItemType| modifies or adds keys.  For example, |\SetItemType{default}{align=flush}| changes the alignment of the built-in |default| type while preserving its other settings.
%
% \begin{example}
% \SetItemType{angle}{
%   align = left,
%   display format = 
%     \textbf{$\langle$#1$\rangle$},
%   ref format     = 
%     $\langle$#1$\rangle$,
%   macro = \angitem
% }
% \begin{prop}
%   \angitem[Angle thesis]
%   Everything is angular.
% \end{prop}
% No further discussion of 
% \Lastref{} is needed.
% \end{example}
%
% \begin{docCommand}{DeclareNumberedType}{\marg{name}\oarg{keys}}
%   Creates a new \LaTeX{} counter named \meta{name} and a matching item type with \texttt{counter=}\meta{name}.  All \refCom{SetItemType} keys are accepted, plus:
% \end{docCommand}
%
% \begin{docKey}{parent}{=\meta{counter}}{no default}
%   A parent counter; the new counter resets when the parent steps (same mechanism as \cs{numberwithin}).  A dedicated \texttt{prop} counter (stepped by each \texttt{prop} and \texttt{inlineprop}) is available for non-persistent numbering.
% \end{docKey}
%
% \begin{docKey}{counter format}{=\meta{format}}{default \meta{name}\texttt{\string\arabic\{\meta{name}\}}}
%   The representation of the new counter (\cs{the}\meta{name}).
% \end{docKey}
%
% \begin{example}
% \DeclareNumberedType{P}
% \begin{prop}
%   \pitem[counter=P] First premise.
%   \label{p1}
%   \pitem[counter=P] Second premise.
%   \label{p2}
% \end{prop}
% From \ref{p1} and \ref{p2}\ldots
% \end{example}
%
% \subsection{Built-in types}
%
% The following item types are predefined.  \refCom{SetItemType} can modify their behaviour.
%
% \begin{center}
% \begin{tabular}{llllll}
%   \textbf{Type} & \textbf{Shortcut} & \textbf{Counter} & \textbf{Display} &
%   \textbf{Ref} & \textbf{Align} \\\hline
%   |plain| & none & none & Name & Name & left \\
%   |default| & none* & none & \textbf{Name} & \textbf{Name} & left \\
%   |long| & |\litem| & none & \textbf{Name} & Name & flush \\
%   |bullet| & |\bitem| & none & \textbullet & \textbullet & default \\
%   |roman| & |\ritem| & |roman| & (i) & (i) & left \\
%   |alph| & |\aitem| & |alph| & (a) & (a) & left \\
%   |numbered| & none$^\dag$ & |numpropi|$^\ddag$  & (1) & (1) & left \\
%   |leveltwo| & none$^\dag$ & |numpropii| & a. & (1a) & left \\
%   |levelthree| & none$^\dag$ & |numpropiii| & (i) & (1a.i) & left \\
%   |levelfour| & none$^\dag$ & |numpropiv| & \textbullet & \textbullet & default \\
%   |levelfive| & none$^\dag$ &  |numpropv| & -- & -- & default
% \end{tabular}
% \end{center}
%
% \begin{itemize}
%   \item[*] The |default| type is auto-selected when |\pitem| or |\ptag| has an optional argument but no |type| key.
%   \item[$\dag$] The |numbered|--|levelfive| types are auto-selected when |\pitem| or |\ptag| has no optional argument, depending on nesting level.
%   \item[$\ddag$] The |equations| package option changes |numbered|'s counter to |equation|.
%     The |numpropii|--|numpropv| counters reset automatically when a |\pitem| at the next lower level is processed.
% \end{itemize}
%
% \section{Cross-referencing}\label{sect:xref}
%
% Labels placed after |\pitem| items work with the standard
% |\label|/|\ref| mechanism.  The key difference from ordinary
% \LaTeX{} references is that |\ref| produces \emph{formatted}
% output: for example, |\textbf| might be applied to the name, or the number might be wrapped in parentheses.  The formatting is controlled by the |format| key (or separately by |display format| and |ref format|).
%
%
% \begin{docCommands}{
%   { doc name = nref,  doc parameter = \marg{label} },
%   { doc name = {nref*}, doc parameter = \marg{label} },
% }
%   ``Naked ref.''  Outputs the bare reference content with all formatting stripped.  If \cs{ref}\texttt{\{premise\}} produces~`(P1)', then \cs{nref}\texttt{\{premise\}} produces~`P1'.  The starred form suppresses the hyperlink.
% \end{docCommands}
%
% |\nref| is often useful in the argument of |\pitem|, when the 
% the name of one proposition should depend on that of another:
% \begin{example}
% \SetItemType{default}{format=(#1)}
% \begin{prop}
%   \pitem[Phys]
%   Everything is physical.
%   \label{premise}
%   \pitem[\nref{premise}*]
%   Almost everything is physical.
%   \label{newpremise}
%   \pitem[\ref{premise}*]
%   The version \ref{newpremise}, which
%   uses |\nref|, looks better 
%   than the one with |\ref|, unless 
%   for some reason one wants two 
%   lots of parentheses.
% \end{prop}
% \end{example}
% 
% \emph{Warning:} documents where the name of one item includes
% a reference to that of another, and there are further 
% references to that item, will require multiple
% \LaTeX\ runs to resolve all references.  To save time, 
% it is better to avoid long chains of dependencies of this sort.
% 
% \begin{docCommands}{
%   { doc name = oref,  doc parameter = \oarg{prefix}\oarg{suffix}\marg{label} },
%   { doc name = {oref*}, doc parameter = \oarg{prefix}\oarg{suffix}\marg{label} },
% }
%   ``Ref with options.''  Extends \cs{ref} by injecting a prefix and/or suffix \emph{inside} the formatting.  With one optional argument, \meta{suffix} is appended; with two, \meta{prefix} is prepended and \meta{suffix} appended.  For instance, if \cs{ref}\texttt{\{premise\}} produces~`(P1)', then \cs{oref}\texttt{[*]\{premise\}} produces~`(P1*)' and \cs{oref}\texttt{[cf.\string~][*]\{premise\}} produces~`(cf.\string~P1*)'.  The starred form suppresses the hyperlink.
% \end{docCommands}
% 
% |\oref| can also be useful in the name of |\pitem|s, if 
% one wants the display format for the modified item
% to depend on that originally used
% \begin{example}
% \begin{prop}
%   \label{premise}
%   \pitem[name=\oref[*]{premise}, 
%      format=#1]
%   This will use boldface and 
%   parentheses because the original
%   referenced item did.
% \end{prop}
% \end{example}
% 
% Another handy use for |\oref| is in combination with |\nref| to refer to ranges:
% \begin{example}
% The first two numbered examples in
% this document were 
% were \oref[--\nref{atomism}]{atoms}.
% \end{example}
%
% 
% \begin{docCommand}{Lastref}{\oarg{prefix}\marg{suffix}}
%   Formatted reference to the most recently processed \cs{pitem} or \cs{ptag}, even without a \cs{label}.  Useful for back-references in running text.  With one argument, \meta{suffix} is appended; with two, \meta{prefix} is also prepended.  Use \cs{Lastref}\texttt{\{\}} for a plain reference.
% \end{docCommand}
%
% \begin{docCommand}{nLastref}{}
%   Like \refCom{Lastref}\texttt{\{\}} but returns the bare content without formatting.  Takes no arguments; simply output any desired suffix directly afterwards.
% \end{docCommand}
%
% \begin{docCommand}{Parentref}{\oarg{prefix}\marg{suffix}}
%   Inside a nested \texttt{prop} (or \texttt{inlineprop}), produces a formatted reference to the most recent item of the enclosing level.  Same argument convention as \refCom{Lastref}.
% \end{docCommand}
%
% \begin{docCommand}{nParentref}{}
%   Like \refCom{Parentref}\texttt{\{\}} but returns the bare content without formatting.  Takes no arguments; simply output any desired suffix directly afterwards.
% \end{docCommand}
%
% \refCom{Parentref} and \refCom{nParentref} are useful for making subitems whose names derive from their parent's:
% \begin{example}
% \DeclareNumberedType{inner}[
%   counter format=\alph{inner},
%   display format=#1.]
% \begin{prop}
%   \pitem[OI] Outer item.
%   \label{outer2}
%   \begin{prop}
%     \pitem[type=inner,
%       format=\Parentref{.#1}]
%     \label{dsub1}
%     Ref: \ref{dsub1},
%     naked: \nref{dsub1}.
%     \pitem[type=inner,
%       display format=#1.,
%       ref format=\Parentref{.#1}]
%     \label{dsub2}
%     Ref: \ref{dsub2},
%     naked: \nref{dsub2}.
%   \end{prop}
% \end{prop}
% \end{example}
%
% The built-in |leveltwo| and |levelthree| types have
% |ref format=\Parentref{#1}| and |ref format=\Parentref{.#1}|,
% respectively, so that if the parent references
% as~`(\textbf{P1})', a |leveltwo| sub-item references
% as~`(\textbf{P1a})' and |\nref| returns just~`a'.
%
% \subsection{How it works: \cs{propapply}}
%
% \begin{docCommand}{propapply}{\marg{template}\marg{content}}
%   Internally, each reference is stored in the \texttt{.aux} file as \cs{propapply}\marg{template}\marg{content}.  The \meta{template} contains formatting with the placeholder \refCom{propfmtarg} where content appears.  At reference time, \cs{propapply} evaluates the template with \cs{propfmtarg} bound to \meta{content}.  The \refCom{oref} and \refCom{nref} commands work by locally redefining \cs{propapply}.
%
%   In normal use, you need not interact with \cs{propapply} directly.
% \end{docCommand}
%
% \begin{docCommand}{propfmtarg}{}
%   Placeholder used inside templates; expands to the content argument of the enclosing \refCom{propapply}.
% \end{docCommand}
%
% \section{Package options}\label{sect:global}
%
% \begin{docCommand}{propoptions}{\marg{keys}}
%   Sets package-level keys.  These can also be set:
%   \begin{itemize}
%     \item In the optional argument of \texttt{prop} and \texttt{inlineprop} (local to that environment).
%     \item In the preamble with \cs{usepackage}\texttt{[}\meta{options}\texttt{]\{propositions\}} (global).
%   \end{itemize}
% \end{docCommand}
% 
%   \emph{Exception 1:} keys containing \texttt{\#} (such as \texttt{equation format}) cannot be set in the optional argument of \cs{usepackage}, due to how \LaTeX{} handles \texttt{\#} in option values.
%
%   \emph{Exception 2:} \texttt{equations} is a global key; it cannot be used in the optional argument of \texttt{prop} or \texttt{inlineprop}.
% 
% \subsection{List dimensions}
%
% \begin{docKeys}[
%     doc parameter = {=\meta{length / length list}},
%   ]
%   {
%     { doc name = topsep },
%     { doc name = partopsep },
%     { doc name = itemsep },
%     { doc name = parsep },
%     { doc name = leftmargin },
%     { doc name = rightmargin },
%     { doc name = labelwidth },
%     { doc name = labelsep },
%     { doc name = itemindent },
%     { doc name = listparindent },
%   }
%   Override the standard \LaTeX{} list dimensions.  Accept the same values as \cs{setlength}, including rubber lengths (e.g.\ \texttt{itemsep=4pt plus 2pt}).  Each key may also take a comma-separated list of per-level values: e.g.\ \texttt{leftmargin=\{2.5em, 0em\}}.  The first value applies at level~1, the second at level~2, etc.  Gaps in the list (e.g.\ \texttt{leftmargin=\{, 0em\}}) cause the class default to be used for that level.
% \end{docKeys}
%
% \begin{docKey}{labelindent}{=\meta{length / length list}}{no default}
%   Positions label left edges at \meta{length} from the enclosing margin, adjusting \cs{labelsep} or \cs{itemindent} as needed.  Positive values move rightward, negative leftward.
% \end{docKey}
%
% \begin{docKey}{tightspacing}{}{no value}
%   Sets all vertical spacing to the compact defaults that the standard document classes use for level-three lists (\texttt{topsep} and \texttt{itemsep} to \texttt{2pt} with stretch/shrink, \texttt{parsep} to \texttt{0pt}, \texttt{partopsep} to \texttt{1pt}).  Individual dimension keys set afterward override.
% \end{docKey}
%
% \begin{docKey}{nosep}{}{no value}
%   Sets \cs{topsep}, \cs{itemsep}, and \cs{parsep} all to zero.
% \end{docKey}
%
% \subsection{Default types}
%
% \begin{docKey}{default type}{=\meta{type}}{initially \texttt{default}}
%   The type used when \cs{pitem} or \cs{ptag} is given a name or counter but no explicit \texttt{type}.
% \end{docKey}
%
% \begin{docKey}{default ptag type}{=\meta{type}}{initially empty}
%   If set, overrides \texttt{default type} for \cs{ptag} only.
% \end{docKey}
%
% \begin{docKey}{level n}{=\meta{type}}{see below}
%   Default item type at nesting level \textit{n} (1--5) when \cs{pitem} has no optional argument.
%   \begin{center}
%     \begin{tabular}{cl}
%       \textbf{Level} & \textbf{Default type} \\
%       1 & \texttt{numbered}\\
%       2 & \texttt{leveltwo}  \\
%       3 & \texttt{levelthree} \\
%       4 & \texttt{levelfour} \\
%       5 & \texttt{levelfive}
%     \end{tabular}
%   \end{center}
%   The \texttt{leveltwo}--\texttt{levelfive} types use special counters \texttt{numpropii}--\texttt{numpropv}, which reset automatically when \texttt{pitem} is used at lesser nesting levels.  Other types can also use these counters.
% \end{docKey}
%
% \subsection{Formatting and referencing equation numbers}
%
% \begin{docKey}{equations}{}{no value, \textbf{global only}}
%   Shares the \texttt{equation} counter between \cs{pitem}\texttt{[type=numbered]} and standard displayed equations.  Also installs the equation format hooks (see below), so that \cs{oref} and \cs{nref} work with equation labels.
% \end{docKey}
%
% \begin{docKey}{equation format}{=\meta{template}}{initially \texttt{(\#1)}}
%   Shorthand: sets both \texttt{equation display format} and \texttt{equation ref format}.  Also installs the equation format hooks independently of the \texttt{equations} option.
% \end{docKey}
%
% \begin{docKey}{equation display format}{=\meta{template}}{initially \texttt{(\#1)}}
%   Controls how equation tags appear in the PDF (via \cs{tagform@}).  Use \texttt{\#1} for the number.  Does not affect \cs{ref} output.  Also installs the equation format hooks independently of the \texttt{equations} option.  Locally scoped.
% \end{docKey}
%
% \begin{docKey}{equation ref format}{=\meta{template}}{initially \texttt{(\#1)}}
%   Controls how \cs{ref} (and \cs{oref}, \cs{nref}) render equation labels.  Use \texttt{\#1} for the number.  Does not affect the displayed tag.  Also installs the equation format hooks independently of the \texttt{equations} option.  Locally scoped.
% \end{docKey}
%
% \section{Compatibility}
%
% The \textsf{propositions} package is designed to work with \textsf{hyperref}, \textsf{cleveref}, and \textsf{amsmath}.
% \textsf{amsmath} is required for |\ptag| and the |equations| option.  With \textsf{cleveref} loaded, all |\pitem|s are assigned to a default |proposition| reference type; the |crefname| key can override this.
%
% Recommended load order:
% \begin{Verbatim}[gobble=2]
% \usepackage{hyperref}
% \usepackage{amsmath}   % if using \ptag
% \usepackage{propositions}
% \usepackage{cleveref}  % if used
% \end{Verbatim}
%
% \section{Known issues}
%
% When using |\ptag| with a named counter (e.g.\ |\ptag[counter=P]|)
% inside an \textsf{amsmath} equation environment, \textsf{hyperref}
% may emit warnings of the form:
% \begin{Verbatim}[gobble=2]
% pdfTeX warning: destination with the same
% identifier (name{equation.N}) has been already
% used, duplicate ignored
% \end{Verbatim}
% These warnings are harmless and do not affect the correctness
% of cross-references.
%
% \StopEventually{}
%
% \section{Implementation}
%
%    \begin{macrocode}
%<*package>
%<doc>%% propositions.sty --- A flexible system for labelling and cross-referencing displayed propositions.
\ProvidesExplPackage {propositions} {2026/02/13} {0.2}
  {Proposition labeling with key-value interface}

\RequirePackage { calc }

%<doc>%% ====================================================================
%<doc>%% Internal variables
%<doc>%% ====================================================================

%<doc>%% --- Nesting level ---
\int_new:N  \g__props_level_int

%<doc>%% --- Per-item state (set during \pitem / \ptag processing) ---
%<doc>%% These are cleared at the start of \__props_resolve_item:n and then
%<doc>%% populated by key parsing and type loading.  Emptiness of a tl
%<doc>%% indicates the key was not set (no separate boolean tracking needed).
\tl_new:N   \l__props_name_tl
\tl_new:N   \l__props_type_tl
\tl_new:N   \l__props_counter_tl
\tl_new:N   \l__props_alignment_tl
\tl_new:N   \l__props_ref_tl
\tl_new:N   \l__props_label_tl
\tl_new:N   \l__props_shorthand_tl
\tl_new:N   \l__props_gloss_tl

%<doc>%% Format functions (one-argument macros set by keys / type loading).
%<doc>%% These always have a definition (defaulting to identity or a standard format),
%<doc>%% so unlike the tl variables above, their "was set" status cannot be inferred
%<doc>%% from their value.
\cs_new:Npn \l__props_displayfmt:n     #1 { #1 }
\cs_new:Npn \l__props_reffmt:n      #1 { #1 }
\cs_new:Npn \l__props_shorthandfmt:n #1 { ~ [#1] }
\cs_new:Npn \l__props_glossfmt:n    #1 { ~ (#1) }

%<doc>%% --- Environment mode ---
%<doc>%% True inside a prop environment, false inside inlineprop.
%<doc>%% Controls whether \pitem outputs a display label or inline text.
\bool_new:N \g__props_display_mode_bool

%<doc>%% --- Global settings ---
\tl_new:N   \l__props_default_type_tl      %% default type (for \pitem, and \ptag fallback)
\tl_new:N   \l__props_default_ptag_type_tl %% \ptag override; empty means use default type
\prop_new:N \l__props_level_defaults_prop  %% nesting level -> default type name
\bool_new:N \g__props_equations_bool         %% true if equations package option is set
\bool_new:N \g__props_eqhooks_bool           %% true if equation format hooks should be installed
\cs_new:Npn \__props_eqdispfmt:n #1 { (#1) } %% display format for equation tags
\cs_new:Npn \__props_eqreffmt:n  #1 { (#1) } %% ref format for equation labels

%<doc>%% --- Working variables ---
\tl_new:N   \l__props_display_text_tl    %% what appears in the document
\tl_new:N   \l__props_ref_text_tl        %% what goes into \@currentlabel
\tl_new:N   \l__props_item_output_tl     %% formatted display (displayfmt + shorthand + gloss)
\bool_new:N \l__props_ptag_bool          %% true during \ptag processing

%<doc>%% --- Parent-ref stacks and last-item storage ---
%<doc>%% Two global seq stacks hold the template/content of the parent item.
%<doc>%% Pushed on \begin{prop}/\begin{inlineprop}, popped on \end.
%<doc>%% Two global tl variables hold the most recent item's template/content,
%<doc>%% used by \Lastref/\nLastref for linguex-style back-references.
\seq_new:N \g__props_parent_tpl_seq
\seq_new:N \g__props_parent_cnt_seq
\tl_new:N  \g__props_last_tpl_tl
\tl_new:N  \g__props_last_cnt_tl

%<doc>%% --- Anchor counter for hyperref ---
\newcounter { propanchor }
%<doc>%% Make hyperref destinations deterministic (not dependent on Hy@linkcounter,
%<doc>%% which amsmath does not save/restore between measuring and output passes).
%<doc>%% Note: hyperref prepends "propanchor." automatically in \H@refstepcounter,
%<doc>%% so \theHpropanchor only needs the unique suffix.
\def \theHpropanchor { \the\value{propanchor} }

%<doc>%% --- Prop environment counter ---
%<doc>%% Stepped by each \begin{prop} / \begin{inlineprop}.  Not displayed;
%<doc>%% exists solely so that \numberwithin{equation}{prop} can reset the
%<doc>%% equation counter per prop block if desired.
\newcounter { prop }

%<doc>%% ====================================================================
%<doc>%% \propapply — format/content separation for cross-references
%<doc>%% ====================================================================
%<doc>%%
%<doc>%% \propfmtarg is a protected placeholder.  In the aux file, format
%<doc>%% templates contain \propfmtarg where the content should go.
%<doc>%% \propapply{<template>}{<content>} evaluates the template with
%<doc>%% \propfmtarg set (via \protected@edef) to the content.  The
%<doc>%% \protected@edef expands expandable macros (including \propfmtarg
%<doc>%% tokens from an outer \propapply scope) while preserving robust
%<doc>%% commands like \textbullet.  This enables correct resolution of
%<doc>%% nested \Parentref references.  \oref and \nref work by locally
%<doc>%% redefining \propapply to inject prefixes/suffixes or strip
%<doc>%% formatting.
%<doc>%%
%<doc>%% When hyperref is loaded, \propapply locally disables hyperref's
%<doc>%% link commands to suppress nested hyperlinks.  This matters because
%<doc>%% ref texts stored in the aux file may contain \ref, \oref, etc.;
%<doc>%% when \ref{item} renders the stored \propapply{template}{content},
%<doc>%% the entire output is already inside a hyperlink created by \ref,
%<doc>%% so any link-generating commands inside must be suppressed.
%<doc>%% (We avoid \NoHyper/\endNoHyper because they use \global
%<doc>%% assignments that break when \propapply is nested.)

\cs_new_protected:Npn \propfmtarg { }

\cs_new_protected:Npn \propapply #1#2
  {
    \group_begin:
    \protected@edef \propfmtarg { #2 }
    #1
    \group_end:
  }

%<doc>%% ====================================================================
%<doc>%% \Parentref / \nParentref / \Lastref / \nLastref
%<doc>%% ====================================================================
%<doc>%%
%<doc>%% \Parentref reads the top of the parent stacks, wrapping the result
%<doc>%% in \propapply so the parent's template is preserved.  When \nref
%<doc>%% strips the outermost \propapply, the parent contribution vanishes,
%<doc>%% leaving only the child content — as expected.
%<doc>%%
%<doc>%% \nParentref returns just the parent's content (bare, no template),
%<doc>%% suitable for use in \the<counter> definitions where the parent
%<doc>%% prefix should be part of the counter *name* and survive \nref.
%<doc>%%
%<doc>%% \Lastref / \nLastref are analogous but read from the last-item
%<doc>%% globals instead of the parent stacks, enabling linguex-style
%<doc>%% back-references without \label.
%<doc>%%
%<doc>%% All four commands are expandable, so they resolve correctly
%<doc>%% inside \protected@edef (e.g. when building aux-file content).
%<doc>%%
%<doc>%% Argument convention:
%<doc>%%   \Parentref{suffix}           — appends suffix to parent content
%<doc>%%   \Parentref[prefix]{suffix}   — wraps parent content with both
%<doc>%%   \Parentref{}                 — plain parent ref (empty suffix)
%<doc>%%
%<doc>%% Note: the last argument must be mandatory for expandable commands
%<doc>%% (xparse requirement), so the suffix is mandatory rather than optional.

\NewExpandableDocumentCommand \Parentref { o m }
  {
    \IfValueTF { #1 }
      {
        \propapply
          { \seq_item:Nn \g__props_parent_tpl_seq { 1 } }
          { #1 \seq_item:Nn \g__props_parent_cnt_seq { 1 } #2 }
      }
      {
        \propapply
          { \seq_item:Nn \g__props_parent_tpl_seq { 1 } }
          { \seq_item:Nn \g__props_parent_cnt_seq { 1 } #2 }
      }
  }

\NewExpandableDocumentCommand \nParentref { }
  { \seq_item:Nn \g__props_parent_cnt_seq { 1 } }

\NewExpandableDocumentCommand \Lastref { o m }
  {
    \IfValueTF { #1 }
      {
        \propapply
          { \g__props_last_tpl_tl }
          { #1 \g__props_last_cnt_tl #2 }
      }
      {
        \propapply
          { \g__props_last_tpl_tl }
          { \g__props_last_cnt_tl #2 }
      }
  }

\NewExpandableDocumentCommand \nLastref { }
  { \g__props_last_cnt_tl }

%<doc>%% ====================================================================
%<doc>%% Variant generation
%<doc>%% ====================================================================

\cs_generate_variant:Nn \tl_gset_eq:NN { cN }
\cs_generate_variant:Nn \cs_gset_eq:NN { cN }
\cs_generate_variant:Nn \cs_set_eq:NN  { Nc }
\cs_generate_variant:Nn \seq_gpush:Nn  { NV }

%<doc>%% ====================================================================
%<doc>%% Type preset storage
%<doc>%% ====================================================================
%<doc>%% For each declared type <T>, we store:
%<doc>%%   \g__props_type_<T>_alignment_tl    (token list)
%<doc>%%   \g__props_type_<T>_counter_tl      (token list)
%<doc>%%   \g__props_type_<T>_displayfmt:n       (one-arg macro)
%<doc>%%   \g__props_type_<T>_reffmt:n        (one-arg macro)
%<doc>%% etc.

\cs_new_protected:Nn \__props_type_ensure:n
  {
    \tl_if_exist:cF { g__props_type_ #1 _alignment_tl }
      {
        \tl_new:c { g__props_type_ #1 _alignment_tl }
        \tl_new:c { g__props_type_ #1 _counter_tl }
        \cs_gset:cpn { g__props_type_ #1 _displayfmt:n }     ##1 { ##1 }
        \cs_gset:cpn { g__props_type_ #1 _reffmt:n }      ##1 { ##1 }
        \cs_gset:cpn { g__props_type_ #1 _shorthandfmt:n }  ##1 { ~ [##1] }
        \cs_gset:cpn { g__props_type_ #1 _glossfmt:n }     ##1 { ~ (##1) }
        \tl_new:c { g__props_type_ #1 _name_tl }
        \tl_new:c { g__props_type_ #1 _ref_tl }
        \tl_new:c { g__props_type_ #1 _shorthand_tl }
        \tl_new:c { g__props_type_ #1 _gloss_tl }
      }
  }

%<doc>%% Load a type's preset into the local variables.
\cs_new_protected:Nn \__props_type_load:
  {
    \tl_if_exist:cTF { g__props_type_ \l__props_type_tl _alignment_tl }
      {
        \tl_set_eq:Nc \l__props_alignment_tl
          { g__props_type_ \l__props_type_tl _alignment_tl }
        \tl_set_eq:Nc \l__props_counter_tl
          { g__props_type_ \l__props_type_tl _counter_tl }
        \cs_set_eq:Nc \l__props_displayfmt:n
          { g__props_type_ \l__props_type_tl _displayfmt:n }
        \cs_set_eq:Nc \l__props_reffmt:n
          { g__props_type_ \l__props_type_tl _reffmt:n }
        \cs_set_eq:Nc \l__props_shorthandfmt:n
          { g__props_type_ \l__props_type_tl _shorthandfmt:n }
        \cs_set_eq:Nc \l__props_glossfmt:n
          { g__props_type_ \l__props_type_tl _glossfmt:n }
        \tl_set_eq:Nc \l__props_name_tl
          { g__props_type_ \l__props_type_tl _name_tl }
        \tl_set_eq:Nc \l__props_ref_tl
          { g__props_type_ \l__props_type_tl _ref_tl }
        \tl_set_eq:Nc \l__props_shorthand_tl
          { g__props_type_ \l__props_type_tl _shorthand_tl }
        \tl_set_eq:Nc \l__props_gloss_tl
          { g__props_type_ \l__props_type_tl _gloss_tl }
        \tl_set_eq:Nc \l__props_crefname_tl
          { g__props_type_ \l__props_type_tl _crefname_tl }
      }
      {
        \msg_error:nnx { props } { unknown-type } { \l__props_type_tl }
      }
  }

\msg_new:nnn { props } { unknown-type }
  { Unknown~ item~ type~ '#1'. }
\msg_new:nnn { props } { unknown-counter }
  { Counter~ '#1'~ not~ defined.~
    Use~ \token_to_str:N \DeclareNumberedType \{ #1 \} ~ in~ preamble. }
\msg_new:nnn { props } { pitem-outside-env }
  { \token_to_str:N \pitem \ can~ only~ be~ used~
    inside~ prop~ or~ inlineprop~ environments. }
\msg_new:nnn { props } { pitem-in-math }
  { \token_to_str:N \pitem \ cannot~ be~ used~ in~ math~ mode.~
    Use~ \token_to_str:N \ptag \ instead. }
\msg_new:nnn { props } { global-only-keys }
  { Key(s)~ '#1'~ ignored~ in~ prop~ optional~ argument~
    (use~ \token_to_str:N \propoptions \ instead). }

%<doc>%% ====================================================================
%<doc>%% l3keys: keys for \SetItemType
%<doc>%% ====================================================================

\tl_new:N   \l__props_decl_macro_tl

\keys_define:nn { props / declare-type }
  {
    name            .tl_set:N  = \l__props_name_tl ,
    align           .tl_set:N  = \l__props_alignment_tl ,
    counter         .tl_set:N  = \l__props_counter_tl ,
    ref             .tl_set:N  = \l__props_ref_tl ,
    display~format  .cs_set:Np = \l__props_displayfmt:n #1 ,
    ref~format      .cs_set:Np = \l__props_reffmt:n #1 ,
    format          .code:n    =
      {
        \cs_set:Npn \l__props_displayfmt:n ##1 { #1 }
        \cs_set:Npn \l__props_reffmt:n ##1 { #1 }
      } ,
    ref~append      .code:n    =
      { \cs_set:Npn \l__props_reffmt:n ##1 { \Parentref{#1} } } ,
    shorthand       .tl_set:N  = \l__props_shorthand_tl ,
    shorthand~format .cs_set:Np = \l__props_shorthandfmt:n #1 ,
    gloss           .tl_set:N  = \l__props_gloss_tl ,
    gloss~format    .cs_set:Np = \l__props_glossfmt:n #1 ,
    crefname        .tl_set:N  = \l__props_crefname_tl ,
    macro           .tl_set:N  = \l__props_decl_macro_tl ,
  }

%<doc>%% ====================================================================
%<doc>%% Helper: define a macro command for a type
%<doc>%% ====================================================================

\cs_new_protected:Nn \__props_define_macro:Nn
  {
    %% #1 = control sequence (e.g. \litem), #2 = type name
    \DeclareDocumentCommand #1 { o }
      {
        \IfValueTF { ##1 }
          {
            %% If the argument has no =, treat it as a bare name
            %% so catcodes (e.g. $math$) are preserved.
            \tl_if_in:nnTF { ##1 } { = }
              { \pitem [ type=#2, ##1 ] }
              { \pitem [ type=#2, name={##1} ] }
          }
          { \pitem [ type=#2 ] }
      }
  }

%<doc>%% ====================================================================
%<doc>%% Helper: store type preset from local state
%<doc>%% ====================================================================

\cs_new_protected:Nn \__props_store_type:n
  {
    \__props_type_ensure:n { #1 }
    \tl_gset_eq:cN { g__props_type_ #1 _alignment_tl }    \l__props_alignment_tl
    \tl_gset_eq:cN { g__props_type_ #1 _counter_tl }      \l__props_counter_tl
    \cs_gset_eq:cN { g__props_type_ #1 _displayfmt:n }       \l__props_displayfmt:n
    \cs_gset_eq:cN { g__props_type_ #1 _reffmt:n }        \l__props_reffmt:n
    \cs_gset_eq:cN { g__props_type_ #1 _shorthandfmt:n }   \l__props_shorthandfmt:n
    \cs_gset_eq:cN { g__props_type_ #1 _glossfmt:n }      \l__props_glossfmt:n
    \tl_gset_eq:cN { g__props_type_ #1 _name_tl }         \l__props_name_tl
    \tl_gset_eq:cN { g__props_type_ #1 _ref_tl }          \l__props_ref_tl
    \tl_gset_eq:cN { g__props_type_ #1 _shorthand_tl }   \l__props_shorthand_tl
    \tl_gset_eq:cN { g__props_type_ #1 _gloss_tl }       \l__props_gloss_tl
    \tl_gset_eq:cN { g__props_type_ #1 _crefname_tl }   \l__props_crefname_tl
  }

%<doc>%% ====================================================================
%<doc>%% Helper: reset local state to defaults for type declaration
%<doc>%% ====================================================================

%<doc>%% Reset local state to defaults for a type declaration.
%<doc>%% counter_tl = "none" is the sentinel for text (non-numbered) items.
%<doc>%% In the per-item reset (\__props_resolve_item:n), counter_tl is cleared
%<doc>%% instead, because there it represents "user didn't specify a counter";
%<doc>%% the actual value is loaded from the type preset in Phase 2.
\cs_new_protected:Nn \__props_decl_reset:
  {
    \tl_set:Nn  \l__props_alignment_tl { default }
    \tl_set:Nn  \l__props_counter_tl   { none }
    \cs_set:Npn \l__props_displayfmt:n ##1     { ##1 }
    \cs_set:Npn \l__props_reffmt:n  ##1     { ##1 }
    \cs_set:Npn \l__props_shorthandfmt:n ##1 { ~ [##1] }
    \cs_set:Npn \l__props_glossfmt:n ##1    { ~ (##1) }
    \tl_clear:N \l__props_name_tl
    \tl_clear:N \l__props_ref_tl
    \tl_clear:N \l__props_shorthand_tl
    \tl_clear:N \l__props_gloss_tl
    \tl_clear:N \l__props_crefname_tl
    \tl_clear:N \l__props_decl_macro_tl
  }

%<doc>%% ====================================================================
%<doc>%% \SetItemType{<name>}{<key=value>}
%<doc>%% ====================================================================

\NewDocumentCommand \SetItemType { m m }
  {
    \__props_decl_reset:
    %% If type already exists, load its preset as the starting point
    %% so that unspecified keys keep their existing values.
    \tl_if_exist:cT { g__props_type_ #1 _alignment_tl }
      {
        \tl_set:Nn \l__props_type_tl { #1 }
        \__props_type_load:
      }
    \keys_set:nn { props / declare-type } { #2 }
    \__props_store_type:n { #1 }
    \tl_if_empty:NF \l__props_decl_macro_tl
      { \exp_args:NV \__props_define_macro:Nn \l__props_decl_macro_tl { #1 } }
  }

%<doc>%% ====================================================================
%<doc>%% l3keys: keys for \DeclareNumberedType
%<doc>%% ====================================================================

\tl_new:N \l__props_numdecl_parent_tl
\tl_new:N \l__props_numdecl_counter_format_tl
%<doc>%% Tracks whether counter format was explicitly given.
%<doc>%% Needed because the default \the<counter> definition differs.
\bool_new:N \l__props_numdecl_has_counter_format_bool

\keys_define:nn { props / declare-numbered }
  {
    %% Counter-specific keys
    parent          .tl_set:N  = \l__props_numdecl_parent_tl ,
    counter~format  .code:n    =
      {
        \tl_set:Nn \l__props_numdecl_counter_format_tl { #1 }
        \bool_set_true:N \l__props_numdecl_has_counter_format_bool
      } ,
    %% Shared keys (same targets as declare-type)
    name            .tl_set:N  = \l__props_name_tl ,
    align           .tl_set:N  = \l__props_alignment_tl ,
    ref             .tl_set:N  = \l__props_ref_tl ,
    display~format  .cs_set:Np = \l__props_displayfmt:n #1 ,
    ref~format      .cs_set:Np = \l__props_reffmt:n #1 ,
    format          .code:n    =
      {
        \cs_set:Npn \l__props_displayfmt:n ##1 { #1 }
        \cs_set:Npn \l__props_reffmt:n ##1 { #1 }
      } ,
    ref~append      .code:n    =
      { \cs_set:Npn \l__props_reffmt:n ##1 { \Parentref{#1} } } ,
    shorthand       .tl_set:N  = \l__props_shorthand_tl ,
    shorthand~format .cs_set:Np = \l__props_shorthandfmt:n #1 ,
    gloss           .tl_set:N  = \l__props_gloss_tl ,
    gloss~format    .cs_set:Np = \l__props_glossfmt:n #1 ,
    macro           .tl_set:N  = \l__props_decl_macro_tl ,
  }

%<doc>%% ====================================================================
%<doc>%% \DeclareNumberedType{<name>}  or  \DeclareNumberedType{<name>}[<keys>]
%<doc>%% ====================================================================

\NewDocumentCommand \DeclareNumberedType { m O{} }
  {
    %% Reset shared state
    \__props_decl_reset:
    %% Override defaults for numbered types
    \cs_set:Npn \l__props_displayfmt:n ##1 { (##1) }
    \cs_set:Npn \l__props_reffmt:n ##1 { (##1) }
    %% Reset counter-specific state
    \tl_clear:N \l__props_numdecl_parent_tl
    \bool_set_false:N \l__props_numdecl_has_counter_format_bool
    %% Process keys
    \keys_set:nn { props / declare-numbered } { #2 }
    %% Create counter
    \newcounter { #1 }
    %% Set parent
    \tl_if_empty:NF \l__props_numdecl_parent_tl
      { \exp_args:NnV \@addtoreset { #1 } \l__props_numdecl_parent_tl }
    %% Define \the<counter>
    \bool_if:NTF \l__props_numdecl_has_counter_format_bool
      { \cs_gset:cpx { the#1 } { \exp_not:V \l__props_numdecl_counter_format_tl } }
      { \cs_gset:cpn { the#1 } { #1 \arabic{#1} } }
    %% Register as item type (force counter to type name)
    \tl_set:Nn \l__props_counter_tl { #1 }
    \__props_store_type:n { #1 }
    %% Define macro command if specified
    \tl_if_empty:NF \l__props_decl_macro_tl
      { \exp_args:NV \__props_define_macro:Nn \l__props_decl_macro_tl { #1 } }
  }

%<doc>%% ====================================================================
%<doc>%% l3keys: keys for \pitem
%<doc>%% ====================================================================

\keys_define:nn { props / pitem }
  {
    name        .tl_set:N = \l__props_name_tl ,
    type        .tl_set:N = \l__props_type_tl ,
    counter     .code:n =
      {
        \tl_set:Nn \l__props_counter_tl { #1 }
        \tl_set:Nn \l__props_name_tl { \use:c { the #1 } }
      } ,
    align       .tl_set:N = \l__props_alignment_tl ,
    ref         .code:n =
      {
        \tl_set:Nn \l__props_ref_tl { #1 }
      } ,
    display~format  .cs_set:Np = \l__props_displayfmt:n #1 ,
    ref~format      .cs_set:Np = \l__props_reffmt:n  #1 ,
    format          .code:n =
      {
        \cs_set:Npn \l__props_displayfmt:n ##1 { #1 }
        \cs_set:Npn \l__props_reffmt:n ##1 { #1 }
      } ,
    ref~append      .code:n =
      { \cs_set:Npn \l__props_reffmt:n ##1 { \Parentref{#1} } } ,
    shorthand       .code:n =
      {
        \tl_set:Nn \l__props_shorthand_tl { #1 }
      } ,
    shorthand~format .cs_set:Np = \l__props_shorthandfmt:n #1 ,
    gloss           .code:n =
      {
        \tl_set:Nn \l__props_gloss_tl { #1 }
      } ,
    gloss~format    .cs_set:Np = \l__props_glossfmt:n #1 ,
    crefname        .tl_set:N = \l__props_crefname_tl ,
    label           .tl_set:N = \l__props_label_tl ,
    unknown     .code:n =
      {
        %% The key NAME is the proposition name (e.g., \pitem[P] -> key "P")
        \tl_set:NV \l__props_name_tl \l_keys_key_str
      } ,
  }

%<doc>%% ====================================================================
%<doc>%% Shared item resolution (used by both \pitem and \ptag)
%<doc>%% ====================================================================

\cs_new_protected:Nn \__props_resolve_item:n
  {
    %% --- Reset per-item state ---
    \tl_clear:N   \l__props_name_tl
    \tl_clear:N   \l__props_type_tl
    \tl_clear:N   \l__props_counter_tl
    \tl_clear:N   \l__props_ref_tl
    \tl_clear:N   \l__props_label_tl
    \tl_clear:N   \l__props_shorthand_tl
    \tl_clear:N   \l__props_gloss_tl
    \tl_clear:N   \l__props_crefname_tl
    \cs_set:Npn \l__props_shorthandfmt:n ##1 { ~ [##1] }
    \cs_set:Npn \l__props_glossfmt:n    ##1 { ~ (##1) }

    %% --- Phase 1: parse user keys ---
    %% If the argument has no = sign, treat it as a bare name so that
    %% catcodes (e.g. $math$) are preserved via the name key's value path
    %% rather than being stringified by the l3keys unknown-key handler.
    \tl_if_empty:nF { #1 }
      {
        \tl_if_in:nnTF { #1 } { = }
          { \keys_set:nn { props / pitem } { #1 } }
          { \keys_set:nn { props / pitem } { name = {#1} } }
      }

    %% --- Resolve type ---
    %% If the user didn't specify type=, infer it:
    %%   - name or counter given -> use the default type (typically "default")
    %%   - nothing given -> use the level default (e.g. "numbered" at level 1)
    \tl_if_empty:NT \l__props_type_tl
      {
        \bool_lazy_or:nnTF
          { ! \tl_if_empty_p:N \l__props_name_tl }
          { ! \tl_if_empty_p:N \l__props_counter_tl }
          {
            %% Name or counter given -> default type.
            %% In ptag mode, use ptag override if set, else fall back.
            \bool_lazy_and:nnTF
              { \l__props_ptag_bool }
              { ! \tl_if_empty_p:N \l__props_default_ptag_type_tl }
              { \tl_set_eq:NN \l__props_type_tl \l__props_default_ptag_type_tl }
              { \tl_set_eq:NN \l__props_type_tl \l__props_default_type_tl }
          }
          {
            %% Nothing specified -> level default
            %% Expand \int_use:N before \prop_get (n-type doesn't expand)
            \tl_set:Nx \l_tmpa_tl { \int_use:N \g__props_level_int }
            \exp_args:NNV \prop_get:NnNTF
              \l__props_level_defaults_prop \l_tmpa_tl \l_tmpb_tl
              { \tl_set_eq:NN \l__props_type_tl \l_tmpb_tl }
              { \tl_set:Nn \l__props_type_tl { numbered } }
          }
      }

    %% --- Phase 2: load type preset (sets counter, formats, etc.) ---
    \__props_type_load:

    %% --- Phase 3: re-apply user keys on top of preset ---
    %% This lets explicit keys like counter= or ref format= override
    %% the values inherited from the type.
    \tl_if_empty:nF { #1 }
      {
        \tl_if_in:nnTF { #1 } { = }
          { \keys_set:nn { props / pitem } { #1 } }
          { \keys_set:nn { props / pitem } { name = {#1} } }
      }

    %% --- Determine display text ---
    %% counter_tl is now the type's counter (or the user's override).
    %% "none" means this is a text item (no counter); anything else is
    %% the name of a LaTeX counter to step and display.
    \tl_if_eq:NnTF \l__props_counter_tl { none }
      {
        %% Text item: display text = name
        \tl_set_eq:NN \l__props_display_text_tl \l__props_name_tl
      }
      {
        %% Numbered item: step counter, display text = \the<counter>
        \cs_if_exist:cTF { c@ \l__props_counter_tl }
          {
            %% In ptag mode inside an equation (not align/gather), the
            %% environment already stepped the equation counter via
            %% \incr@eqnum.  We skip our step to avoid a gap.
            %% In align/gather, no per-row step happens, so we step.
            %% For non-equation counters (e.g. P), always step.
            \bool_if:NTF \l__props_ptag_bool
              {
                %% ptag mode: skip step for equation counter in equation env
                %% (the env already stepped it); step in align/gather or for
                %% non-equation counters.
                \tl_if_eq:NnTF \l__props_counter_tl { equation }
                  {
                    \ifinalign@
                      \stepcounter { \l__props_counter_tl }
                    \else \ifingather@
                      \stepcounter { \l__props_counter_tl }
                    \fi \fi
                  }
                  { \stepcounter { \l__props_counter_tl } }
              }
              { \stepcounter { \l__props_counter_tl } }
            %% If no explicit name, default to \the<counter>
            \tl_if_empty:NT \l__props_name_tl
              {
                \protected@edef \l__props_name_tl
                  { \use:c { the \l__props_counter_tl } }
              }
            \protected@edef \l__props_display_text_tl
              { \l__props_name_tl }
          }
          {
            \msg_error:nnx { props } { unknown-counter }
              { \l__props_counter_tl }
            \tl_set:Nn \l__props_display_text_tl { ??? }
          }
      }

    %% --- Determine ref text ---
    %% Priority: explicit ref > shorthand > display text
    \tl_if_empty:NTF \l__props_ref_tl
      {
        \tl_if_empty:NTF \l__props_shorthand_tl
          { \tl_set_eq:NN \l__props_ref_text_tl \l__props_display_text_tl }
          { \tl_set_eq:NN \l__props_ref_text_tl \l__props_shorthand_tl }
      }
      { \tl_set_eq:NN \l__props_ref_text_tl \l__props_ref_tl }

    %% --- Apply display format to display text ---
    \protected@edef \l__props_item_output_tl
      { \l__props_displayfmt:n { \l__props_display_text_tl } }

    %% --- Append shorthand to display if present ---
    \tl_if_empty:NF \l__props_shorthand_tl
      {
        \protected@edef \l_tmpa_tl
          { \l__props_shorthandfmt:n { \l__props_shorthand_tl } }
        \tl_put_right:NV \l__props_item_output_tl \l_tmpa_tl
      }

    %% --- Append gloss to display if present ---
    \tl_if_empty:NF \l__props_gloss_tl
      {
        \protected@edef \l_tmpa_tl
          { \l__props_glossfmt:n { \l__props_gloss_tl } }
        \tl_put_right:NV \l__props_item_output_tl \l_tmpa_tl
      }
  }

%<doc>%% ====================================================================
%<doc>%% Ref-append helpers (shared by \pitem and \ptag)
%<doc>%% ====================================================================

%<doc>%% ====================================================================
%<doc>%% \pitem — the main item command
%<doc>%% ====================================================================

\NewDocumentCommand \pitem { o }
  {
    \int_compare:nNnF { \g__props_level_int } > { 0 }
      { \msg_error:nn { props } { pitem-outside-env } }
    \mode_if_math:T
      { \msg_error:nn { props } { pitem-in-math } }
    %% Reset the sub-level counter one level below the current level.
    %% This is done in \pitem (not \begin{prop}) so that a plain \item
    %% can skip the reset, enabling intertext between sub-items.
    \int_compare:nNnT { \g__props_level_int } = { 1 }
      { \setcounter { numpropii } { 0 } }
    \int_compare:nNnT { \g__props_level_int } = { 2 }
      { \setcounter { numpropiii } { 0 } }
    \int_compare:nNnT { \g__props_level_int } = { 3 }
      { \setcounter { numpropiv } { 0 } }
    \int_compare:nNnT { \g__props_level_int } = { 4 }
      { \setcounter { numpropv } { 0 } }
    \IfValueTF { #1 }
      { \__props_resolve_item:n { #1 } }
      { \__props_resolve_item:n { } }
    \__props_set_ref:
    \__props_output_item:
    \tl_if_empty:NF \l__props_label_tl
      { \exp_args:NV \label \l__props_label_tl }
  }

%<doc>%% ====================================================================
%<doc>%% Cross-referencing: set \@currentlabel
%<doc>%% ====================================================================

%<doc>%% Build propapply-formatted ref text into \l__props_ref_result_tl.
\tl_new:N \l__props_ref_result_tl

\cs_new_protected:Nn \__props_build_ref_text:
  {
    %% Build a \propapply{template}{content} token list for the aux file.
    %% Also save the template and content globally for \Lastref and the
    %% parent stacks (pushed when entering a nested prop environment).
    \protected@xdef \g__props_last_tpl_tl
      { \l__props_reffmt:n { \propfmtarg } }
    \protected@xdef \g__props_last_cnt_tl
      { \l__props_ref_text_tl }
    \protected@edef \l__props_ref_result_tl
      {
        \propapply
          { \exp_not:V \g__props_last_tpl_tl }
          { \exp_not:V \g__props_last_cnt_tl }
      }
  }

%<doc>%% Override the cleveref type recorded by \cs{refstepcounter}.
%<doc>%% Parses \cs{cref@currentlabel} = |[type]..rest..| and replaces the type.
\cs_new_protected:Npn \__props_override_cref_type:n #1
  {
    \exp_after:wN \__props_replace_cref_type:w
      \cref@currentlabel \q_stop { #1 }
  }
\cs_new_protected:Npn \__props_replace_cref_type:w [ #1 ] #2 \q_stop #3
  { \def \cref@currentlabel { [ #3 ] #2 } }

\cs_new_protected:Nn \__props_set_ref:
  {
    \__props_build_ref_text:
    \cs_set:Npx \thepropanchor { \exp_not:V \l__props_ref_result_tl }
    \refstepcounter { propanchor }
    %% If cleveref is loaded and crefname is set, override the type
    %% so that any subsequent \label picks up the new type.
    \bool_lazy_and:nnT
      { \g__props_cleveref_bool }
      { ! \tl_if_empty_p:N \l__props_crefname_tl }
      { \exp_args:NV \__props_override_cref_type:n \l__props_crefname_tl }
  }

%<doc>%% ====================================================================
%<doc>%% Save original \ref (for \oref / \nref circularity protection)
%<doc>%% ====================================================================

%<doc>%% Must happen at begin-document so that hyperref's redefinition of
%<doc>%% \ref is already in place.  \oref and \nref call this saved copy
%<doc>%% instead of \ref, so \let\ref\oref works without circularity.
\AtBeginDocument
  { \cs_set_eq:NN \__props_orig_ref:w \ref }

%<doc>%% ====================================================================
%<doc>%% Hyperref PDF string support
%<doc>%% ====================================================================

\AtBeginDocument
  {
    \@ifpackageloaded { hyperref }
      {
        %% Upgrade \propapply to suppress nested hyperlinks.
        %% We disable hyperref link commands locally (within \group_begin:
        %% / \group_end:) rather than using \NoHyper / \endNoHyper, because
        %% those operate via \global assignments and break when \propapply
        %% is nested (the inner \NoHyper overwrites the saved live-link).
        \cs_set_protected:Npn \propapply #1#2
          {
            \group_begin:
            \protected@edef \propfmtarg { #2 }
            \def \hyper@link@ [##1]##2##3##4{##4\Hy@xspace@end}
            \def \hyper@@anchor ##1##2{##2\Hy@xspace@end}
            \def \hyper@link ##1##2##3{##3\Hy@xspace@end}
            \let \hyper@anchor \@gobble
            \let \hyper@anchorstart \@gobble
            \def \hyper@anchorend {\Hy@xspace@end}
            \let \hyper@linkstart \@gobbletwo
            \def \hyper@linkend {\Hy@xspace@end}
            \def \hyper@linkurl ##1##2{##1\Hy@xspace@end}
            \def \hyper@linkfile ##1##2##3{##1\Hy@xspace@end}
            \let \Hy@backout \@gobble
            #1
            \group_end:
          }
        %% Make \autoref work for propanchor labels (outputs the
        %% formatted ref text with no extra prefix).
        \def \propanchorautorefname { }
        \pdfstringdefDisableCommands
          {
            \def \propapply #1#2{ #2 }
            \def \propfmtarg { }
            \def \oref #1{ \ref{#1} }
            \def \nref #1{ \ref{#1} }
            \def \Parentref #1{ }
            \def \nParentref { }
            \def \Lastref #1{ }
            \def \nLastref { }
          }
      }
      { }
  }

%<doc>%% ====================================================================
%<doc>%% Cleveref integration
%<doc>%% ====================================================================

\bool_new:N \g__props_cleveref_bool

\AtBeginDocument
  {
    \@ifpackageloaded { cleveref }
      {
        \bool_gset_true:N \g__props_cleveref_bool
        %% Map the propanchor counter type to "proposition"
        \crefalias { propanchor } { proposition }
        %% Empty name: \cref just outputs the formatted ref text
        \crefname  { proposition } {} {}
        \Crefname  { proposition } {} {}
      }
      { }
  }

%<doc>%% ====================================================================
%<doc>%% Equation formatting integration
%<doc>%% ====================================================================

\AtBeginDocument
  {
    \bool_if:NT \g__props_equations_bool
      { \bool_gset_true:N \g__props_eqhooks_bool }
    \bool_if:NT \g__props_eqhooks_bool
      {
        %% Redefine the equation "prefix" macro so that \ref{eq:foo}
        %% produces \propapply{format}{number} instead of a bare number.
        %% This makes equation refs work with \oref and \nref.
        \cs_set:cpn { p@equation } #1
          { \propapply { \__props_eqreffmt:n { \propfmtarg } } { #1 } }
        %% Redefine \tagform@ so that displayed equation tags use
        %% the display format.  \tag* (used by \ptag) bypasses this.
        \cs_set:Npn \tagform@ #1
          { \maketag@@@
              { \__props_eqdispfmt:n
                  { \ignorespaces #1 \unskip \@@italiccorr } } }
      }
  }

%<doc>%% ====================================================================
%<doc>%% \ptag — proposition tag for displayed math environments
%<doc>%% ====================================================================

%<doc>%% Wrapper to call \tag* — needed because \exp_args:NV can't expand
%<doc>%% into a command with * directly.
\cs_new_protected:Nn \__props_tag_star:n { \tag* { #1 } }

%<doc>%% Deferred ref-setting for ptags.  Amsmath typesets equations in two
%<doc>%% passes (measuring + output); we must only set the ref on the output
%<doc>%% pass (\ifmeasuring@ is false) to avoid double-stepping propanchor.
%<doc>%% \#1 = ref text, \#2 = crefname (may be empty).
\cs_new_protected:Nn \__props_ptag_finish_ref:nn
  {
    \ifmeasuring@ \else
    \cs_set:Npx \thepropanchor { \exp_not:n { #1 } }
    \refstepcounter { propanchor }
    \tl_if_empty:nF { #2 }
      {
        \bool_if:NT \g__props_cleveref_bool
          { \__props_override_cref_type:n { #2 } }
      }
    \fi
  }

\NewDocumentCommand \ptag { o }
  {
    %% Enter ptag mode so \__props_resolve_item:n can adjust counter stepping
    \bool_set_true:N \l__props_ptag_bool

    %% Process keys (same resolution as \pitem)
    \IfValueTF { #1 }
      { \__props_resolve_item:n { #1 } }
      { \__props_resolve_item:n { } }

    \bool_set_false:N \l__props_ptag_bool

    %% Build propapply-formatted ref text
    \__props_build_ref_text:

    %% \tag* internally calls \nonumber, which decrements the equation
    %% counter.  When the ptag itself stepped that counter, we must
    %% neutralize \incr@eqnum to prevent the decrement.  For ptags using
    %% other counters (e.g. P), the equation counter should be unaffected.
    \tl_if_eq:NnT \l__props_counter_tl { equation }
      { \cs_set_eq:NN \incr@eqnum \scan_stop: }

    %% Use \tag* for amsmath integration
    \exp_args:NV \__props_tag_star:n \l__props_item_output_tl

    %% Append our ref-setting code to \df@tag.  Amsmath collects all
    %% \tag actions into \df@tag and executes them on the output pass.
    %% We piggyback on this mechanism so our \refstepcounter runs at
    %% the right time (after measuring, not during).
    \cs_gset:Npx \df@tag
      {
        \exp_not:o { \df@tag }
        \exp_not:N \__props_ptag_finish_ref:nn
          { \exp_not:V \l__props_ref_result_tl }
          { \exp_not:V \l__props_crefname_tl }
      }

    %% Handle label key via amsmath's deferred label mechanism
    \tl_if_empty:NF \l__props_label_tl
      { \exp_args:NV \label \l__props_label_tl }
  }

%<doc>%% ====================================================================
%<doc>%% Item output
%<doc>%% ====================================================================

\cs_new_protected:Nn \__props_output_item:
  {
    \bool_if:NTF \g__props_display_mode_bool
      { \__props_output_item_display: }
      { \__props_output_item_inline: }
  }

%<doc>%% Display mode: use \item[...] with alignment control.
%<doc>%%   "flush"   — label starts at the left text margin (pushes text right)
%<doc>%%   "left"    — label is left-aligned in the label box (natural position)
%<doc>%%   "right"   — label is right-aligned in the label box (like enumerate)
%<doc>%%   "default" — standard LaTeX left-aligned label
\cs_new_protected:Nn \__props_output_item_display:
  {
    \str_if_eq:VnTF \l__props_alignment_tl { flush }
      {
        %% Push label past the label area so it sits at the text margin
        \item [ \hspace{\labelwidth} \hspace{\labelsep}
                \l__props_item_output_tl \hspace{\labelsep} ]
      }
      {
        \str_if_eq:VnTF \l__props_alignment_tl { left }
          {
            %% Negative space to left-align in the label box
            \item [ \hspace{\dimexpr \labelwidth + \labelsep
                      - \leftmargin - \itemindent \relax}
                    \l__props_item_output_tl ]
          }
          {
            \str_if_eq:VnTF \l__props_alignment_tl { left-nextline }
              {
                %% Label on own line(s), starting at the left edge
                %% of the label area.  Issue a blank \item and typeset
                %% the label as ordinary paragraph text with negative
                %% \leftskip; then cancel the inter-paragraph space.
                \item []
                { \leftskip = \dimexpr -\labelwidth - \labelsep \relax
                  \l__props_item_output_tl \par }
                \vspace { -\parsep } \noindent \ignorespaces
              }
              {
            \str_if_eq:VnTF \l__props_alignment_tl { nextline }
              {
                %% Label on own line(s), starting at the default label
                %% position (same as alignment=default).
                \item []
                { \leftskip = \dimexpr \itemindent
                    - \labelwidth - \labelsep \relax
                  \l__props_item_output_tl \par }
                \vspace { -\parsep } \noindent \ignorespaces
              }
              {
            \str_if_eq:VnTF \l__props_alignment_tl { flush-nextline }
              {
                %% Label on own line(s), starting at the text margin.
                %% Blank \item, label as ordinary paragraph text,
                %% then cancel the inter-paragraph space.
                \item []
                \l__props_item_output_tl \par
                \vspace { -\parsep } \noindent \ignorespaces
              }
              {
            \str_if_eq:VnTF \l__props_alignment_tl { right }
              {
                %% Right-align in label box (like standard enumerate)
                \cs_set:Nn \__props_display_label:n { \hss \mbox{##1} }
                \item [ \l__props_item_output_tl ]
                \cs_set:Nn \__props_display_label:n { \mbox{##1} \hfill }
              }
              { \item [ \l__props_item_output_tl ] }
              }
              }
              }
          }
      }
  }

%<doc>%% Inline mode: just output text
\cs_new_protected:Nn \__props_output_item_inline:
  { \l__props_item_output_tl }

%<doc>%% ====================================================================
%<doc>%% The prop environment
%<doc>%% ====================================================================

%<doc>%% --- List dimension overrides (empty = use LaTeX default) ---
%<doc>%% These use local assignment so \propoptions in a group is scoped.
\tl_new:N \l__props_topsep_tl
\tl_new:N \l__props_partopsep_tl
\tl_new:N \l__props_itemsep_tl
\tl_new:N \l__props_parsep_tl
\tl_new:N \l__props_leftmargin_tl
\tl_new:N \l__props_labelwidth_tl
\tl_new:N \l__props_rightmargin_tl
\tl_new:N \l__props_labelsep_tl
\tl_new:N \l__props_itemindent_tl
\tl_new:N \l__props_listparindent_tl
\tl_new:N \l__props_labelindent_tl

%<doc>%% \makelabel for prop lists: left-aligns the label and prevents line breaks.
\cs_new:Nn \__props_display_label:n { \mbox{#1} \hfill }

%<doc>%% Helper: apply a dimension override that may be a single value or a
%<doc>%% comma-separated per-level list.  \#1 = tl variable, \#2 = dimension.
%<doc>%% Empty tl $\Rightarrow$ inherit class default.
%<doc>%% Single value $\Rightarrow$ apply to all levels.
%<doc>%% Comma-list $\Rightarrow$ pick the item for the current prop level;
%<doc>%% if the level exceeds the list length, inherit class default.
%<doc>%% Uses \cs{seq\_set\_split:NnV} (not clist) so that blank entries
%<doc>%% are preserved: e.g.\ \texttt{leftmargin=\{, 0em\}} means
%<doc>%% ``class default at level~1, 0em at level~2.''
\seq_new:N \l__props_dim_split_seq
\cs_generate_variant:Nn \seq_set_split:Nnn { NnV }
\cs_new:Nn \__props_apply_dim:Nn
  {
    \tl_if_empty:NF #1
      {
        \seq_set_split:NnV \l__props_dim_split_seq { , } #1
        \int_compare:nNnTF
          { \seq_count:N \l__props_dim_split_seq } > { 1 }
          {
            \tl_set:Nx \l_tmpa_tl
              { \seq_item:Nn \l__props_dim_split_seq
                  { \g__props_level_int } }
            \tl_if_empty:NF \l_tmpa_tl
              { \setlength { #2 } { \l_tmpa_tl } }
          }
          { \setlength { #2 } { #1 } }
      }
  }

\NewDocumentEnvironment { prop } { O{} }
  {
    \stepcounter { prop }
    \int_gincr:N \g__props_level_int
    %% Push last item's template/content onto parent stacks
    \seq_gpush:NV \g__props_parent_tpl_seq \g__props_last_tpl_tl
    \seq_gpush:NV \g__props_parent_cnt_seq \g__props_last_cnt_tl
    \bool_gset_true:N \g__props_display_mode_bool
    %% Apply per-environment overrides; reject global-only keys.
    \keys_set_filter:nnnN { props / global } { global-only }
      { #1 } \l_tmpa_tl
    \tl_if_empty:NF \l_tmpa_tl
      {
        \msg_warning:nnx { props } { global-only-keys }
          { \l_tmpa_tl }
      }
    \begin{list} {}
      {
        \cs_set:Npn \makelabel ##1 { \__props_display_label:n {##1} }
        %% Apply dimension overrides.  Each value may be a single length
        %% (applied at all levels) or a comma-list of per-level lengths.
        %% Empty = inherit the document class default.
        \__props_apply_dim:Nn \l__props_topsep_tl        \topsep
        \__props_apply_dim:Nn \l__props_partopsep_tl      \partopsep
        \__props_apply_dim:Nn \l__props_itemsep_tl        \itemsep
        \__props_apply_dim:Nn \l__props_parsep_tl         \parsep
        \__props_apply_dim:Nn \l__props_leftmargin_tl     \leftmargin
        \__props_apply_dim:Nn \l__props_rightmargin_tl    \rightmargin
        \__props_apply_dim:Nn \l__props_labelsep_tl       \labelsep
        \__props_apply_dim:Nn \l__props_itemindent_tl     \itemindent
        \__props_apply_dim:Nn \l__props_listparindent_tl  \listparindent
        \__props_apply_dim:Nn \l__props_labelwidth_tl     \labelwidth
        %% labelindent: if set, position the label's left edge at
        %% <labelindent> from the enclosing margin by adjusting
        %% labelsep or itemindent (labelwidth is left alone).
        %% Case 1 (label fits): keep itemindent, widen labelsep.
        %% Case 2 (label overflows): keep labelsep, increase itemindent.
        \tl_if_empty:NF \l__props_labelindent_tl
          {
            \seq_set_split:NnV \l__props_dim_split_seq
              { , } \l__props_labelindent_tl
            \int_compare:nNnTF
              { \seq_count:N \l__props_dim_split_seq } > { 1 }
              {
                \tl_set:Nx \l_tmpa_tl
                  { \seq_item:Nn \l__props_dim_split_seq
                      { \g__props_level_int } }
              }
              { \tl_set_eq:NN \l_tmpa_tl \l__props_labelindent_tl }
            \tl_if_empty:NF \l_tmpa_tl
              {
                \dim_compare:nTF
                  { \labelwidth + \l_tmpa_tl <= \leftmargin + \itemindent }
                  { %% Case 1: label fits — absorb slack into labelsep
                    \setlength { \labelsep }
                      { \dimexpr \leftmargin + \itemindent
                          - \labelwidth - \l_tmpa_tl \relax }
                  }
                  { %% Case 2: label overflows — push first line right
                    \setlength { \itemindent }
                      { \dimexpr \labelwidth + \labelsep
                          - \leftmargin + \l_tmpa_tl \relax }
                  }
              }
          }
      }
  }
  {
    \end{list}
    %% Pop parent stacks.  At level >= 2, restore the popped values
    %% into \g__props_last_* so that re-entering a prop for intertext
    %% sees the correct parent (not a stale sub-item).
    \seq_gpop:NN \g__props_parent_tpl_seq \l_tmpa_tl
    \seq_gpop:NN \g__props_parent_cnt_seq \l_tmpb_tl
    %% At level >= 2, restore the popped parent values into
    %% \g__props_last_* so that re-entering a prop for intertext
    %% sees the correct parent (not a stale sub-item).
    \int_compare:nNnT { \g__props_level_int } > { 1 }
      {
        \tl_gset_eq:NN \g__props_last_tpl_tl \l_tmpa_tl
        \tl_gset_eq:NN \g__props_last_cnt_tl \l_tmpb_tl
      }
    \int_gdecr:N \g__props_level_int
    %% Restore display_mode: true if we're still inside an outer prop,
    %% false if we've exited to top level (or into an inlineprop).
    \int_compare:nNnTF { \g__props_level_int } > { 0 }
      { \bool_gset_true:N \g__props_display_mode_bool }
      { \bool_gset_false:N \g__props_display_mode_bool }
  }

%<doc>%% ====================================================================
%<doc>%% The inlineprop environment
%<doc>%% ====================================================================

\NewDocumentEnvironment { inlineprop } { O{} }
  {
    \stepcounter { prop }
    \int_gincr:N \g__props_level_int
    %% Push last item's template/content onto parent stacks
    \seq_gpush:NV \g__props_parent_tpl_seq \g__props_last_tpl_tl
    \seq_gpush:NV \g__props_parent_cnt_seq \g__props_last_cnt_tl
    %% Apply per-environment overrides; reject global-only keys.
    \keys_set_filter:nnnN { props / global } { global-only }
      { #1 } \l_tmpa_tl
    \tl_if_empty:NF \l_tmpa_tl
      {
        \msg_warning:nnx { props } { global-only-keys }
          { \l_tmpa_tl }
      }
  }
  {
    %% Pop parent stacks (same restore logic as prop environment)
    \seq_gpop:NN \g__props_parent_tpl_seq \l_tmpa_tl
    \seq_gpop:NN \g__props_parent_cnt_seq \l_tmpb_tl
    \int_compare:nNnT { \g__props_level_int } > { 1 }
      {
        \tl_gset_eq:NN \g__props_last_tpl_tl \l_tmpa_tl
        \tl_gset_eq:NN \g__props_last_cnt_tl \l_tmpb_tl
      }
    \int_gdecr:N \g__props_level_int
  }

%<doc>%% ====================================================================
%<doc>%% \oref — ref with optional prefix/suffix
%<doc>%% ====================================================================

%<doc>%% Helper: locally redefine \propapply to inject prefix (#1) and
%<doc>%% suffix (#2) into the content slot.  Uses \cs_set_protected:Npx to
%<doc>%% freeze the prefix/suffix values at definition time.
\cs_new_protected:Nn \__props_oref_setup:nn
  {
    \tl_set:Nn \l_tmpa_tl { #1 }
    \tl_set:Nn \l_tmpb_tl { #2 }
    %% Save the current (standard) \propapply so we can restore it
    %% inside the redefined version.  This ensures prefix/suffix only
    %% apply at the outermost \propapply; any nested \propapply from
    %% \Parentref etc. uses the standard definition.
    \cs_set_eq:NN \__props_std_propapply:ww \propapply
    \cs_set_protected:Npx \propapply ##1##2
      {
        \exp_not:N \group_begin:
        \exp_not:N \cs_set_eq:NN \exp_not:N \propapply
          \exp_not:N \__props_std_propapply:ww
        \exp_not:N \protected@edef \exp_not:N \propfmtarg
          { \exp_not:V \l_tmpa_tl ##2 \exp_not:V \l_tmpb_tl }
        ##1
        \exp_not:N \group_end:
      }
  }

\NewDocumentCommand \oref { s o o m }
  {
    \group_begin:
    \IfValueTF { #3 }
      { \__props_oref_setup:nn { #2 } { #3 } }
      {
        \IfValueT { #2 }
          { \__props_oref_setup:nn { } { #2 } }
      }
    \IfBooleanTF { #1 }
      { \__props_orig_ref:w * { #4 } }
      { \__props_orig_ref:w   { #4 } }
    \group_end:
  }

%<doc>%% ====================================================================
%<doc>%% \nref — naked ref (strips formatting)
%<doc>%% ====================================================================

\NewDocumentCommand \nref { s m }
  {
    \group_begin:
    \cs_set:Npn \propapply ##1##2 { ##2 }
    \IfBooleanTF { #1 }
      { \__props_orig_ref:w * { #2 } }
      { \__props_orig_ref:w   { #2 } }
    \group_end:
  }

%<doc>%% ====================================================================
%<doc>%% \propoptions — global configuration
%<doc>%% ====================================================================

\keys_define:nn { props / global }
  {
    default~type      .tl_set:N    = \l__props_default_type_tl ,
    default~ptag~type .tl_set:N    = \l__props_default_ptag_type_tl ,
    equation~format   .code:n      =
      {
        \bool_gset_true:N \g__props_eqhooks_bool
        \cs_set:Npn \__props_eqdispfmt:n ##1 { #1 }
        \cs_set:Npn \__props_eqreffmt:n  ##1 { #1 }
      } ,
    equation~display~format .code:n =
      {
        \bool_gset_true:N \g__props_eqhooks_bool
        \cs_set:Npn \__props_eqdispfmt:n ##1 { #1 }
      } ,
    equation~ref~format     .code:n =
      {
        \bool_gset_true:N \g__props_eqhooks_bool
        \cs_set:Npn \__props_eqreffmt:n  ##1 { #1 }
      } ,
    equations         .bool_gset:N = \g__props_equations_bool ,
    equations         .default:n   = { true } ,
    equations         .groups:n    = { global-only } ,
    tightspacing      .code:n      =
      {
        \tl_set:Nn \l__props_topsep_tl     { 2\p@ \@plus\p@ \@minus\p@ }
        \tl_set:Nn \l__props_itemsep_tl    { 2\p@ \@plus\p@ \@minus\p@ }
        \tl_set:Nn \l__props_parsep_tl     { \z@ }
        \tl_set:Nn \l__props_partopsep_tl  { \p@ \@plus\z@ \@minus\p@ }
      } ,
    nosep             .code:n      =
      {
        \tl_set:Nn \l__props_topsep_tl  { 0pt }
        \tl_set:Nn \l__props_itemsep_tl { 0pt }
        \tl_set:Nn \l__props_parsep_tl  { 0pt }
      } ,
    level~1           .code:n      =
      { \prop_put:Nnn \l__props_level_defaults_prop { 1 } { #1 } } ,
    level~2           .code:n      =
      { \prop_put:Nnn \l__props_level_defaults_prop { 2 } { #1 } } ,
    level~3           .code:n      =
      { \prop_put:Nnn \l__props_level_defaults_prop { 3 } { #1 } } ,
    level~4           .code:n      =
      { \prop_put:Nnn \l__props_level_defaults_prop { 4 } { #1 } } ,
    level~5           .code:n      =
      { \prop_put:Nnn \l__props_level_defaults_prop { 5 } { #1 } } ,
    %% List dimensions
    topsep            .tl_set:N    = \l__props_topsep_tl ,
    partopsep         .tl_set:N    = \l__props_partopsep_tl ,
    itemsep           .tl_set:N    = \l__props_itemsep_tl ,
    parsep            .tl_set:N    = \l__props_parsep_tl ,
    leftmargin        .tl_set:N    = \l__props_leftmargin_tl ,
    rightmargin       .tl_set:N    = \l__props_rightmargin_tl ,
    labelwidth        .tl_set:N    = \l__props_labelwidth_tl ,
    labelsep          .tl_set:N    = \l__props_labelsep_tl ,
    itemindent        .tl_set:N    = \l__props_itemindent_tl ,
    listparindent     .tl_set:N    = \l__props_listparindent_tl ,
    labelindent       .tl_set:N    = \l__props_labelindent_tl ,
  }

\NewDocumentCommand \propoptions { m }
  { \keys_set:nn { props / global } { #1 } }

%<doc>%% ====================================================================
%<doc>%% Default global settings (set before options so user options override)
%<doc>%% ====================================================================

\tl_set:Nn \l__props_default_type_tl { default }
\prop_put:Nnn \l__props_level_defaults_prop { 1 } { numbered }
\prop_put:Nnn \l__props_level_defaults_prop { 2 } { leveltwo }
\prop_put:Nnn \l__props_level_defaults_prop { 3 } { levelthree }
\prop_put:Nnn \l__props_level_defaults_prop { 4 } { levelfour }
\prop_put:Nnn \l__props_level_defaults_prop { 5 } { levelfive }

%<doc>%% ====================================================================
%<doc>%% Package options (processed via l3keys)
%<doc>%% ====================================================================

\ProcessKeyOptions [ props / global ]

%<doc>%% ====================================================================
%<doc>%% Default type declarations
%<doc>%% ====================================================================

\SetItemType { plain }
  {
    counter     = none ,
    format      = #1 ,
  }

\SetItemType { default }
  {
    counter     = none ,
    format      = \textbf{#1} ,
  }

\SetItemType { long }
  {
    align          = flush ,
    counter        = none ,
    display~format = \textbf{#1} ,
    ref~format     = #1 ,
    macro          = \litem ,
  }

\SetItemType { bullet }
  {
    align          = default ,
    name           = \textbullet ,
    display~format = #1 ,
    macro          = \bitem ,
  }

\bool_if:NTF \g__props_equations_bool
  {
    \SetItemType { numbered }
      {
        align          = left ,
        counter        = equation ,
        display~format = \__props_eqdispfmt:n{#1} ,
        ref~format     = \__props_eqreffmt:n{#1} ,
      }
  }
  {
    \newcounter { numpropi }
    \SetItemType { numbered }
      {
        align       = left ,
        counter     = numpropi ,
        format      = (#1) ,
      }
  }

%<doc>%% ====================================================================
%<doc>%% Default numbered types
%<doc>%% ====================================================================

\DeclareNumberedType { roman }
  [
    parent         = section ,
    counter~format = \roman{roman} ,
    format         = (#1) ,
    macro          = \ritem ,
  ]

\DeclareNumberedType { alph }
  [
    parent         = section ,
    counter~format = \alph{alph} ,
    format         = (#1) ,
    macro          = \aitem ,
  ]

\newcounter { numpropii }
\cs_gset:Npn \thenumpropii { \alph { numpropii } }
\SetItemType { leveltwo }
  {
    counter        = numpropii ,
    display~format = #1. ,
    ref~format     = \Parentref{#1} ,
  }

\newcounter { numpropiii }
\cs_gset:Npn \thenumpropiii { \roman { numpropiii } }
\SetItemType { levelthree }
  {
    counter        = numpropiii ,
    display~format = (#1) ,
    ref~format     = \Parentref{.#1} ,
  }

\newcounter { numpropiv }
\SetItemType { levelfour }
  {
    counter        = numpropiv ,
    align          = default ,
    name           = \textbullet ,
    display~format = #1 ,
  }

\newcounter { numpropv }
\SetItemType { levelfive }
  {
    counter        = numpropv ,
    align          = default ,
    name           = -- ,
    display~format = #1 ,
  }

%    \end{macrocode}
%</package>
%
% \Finale
