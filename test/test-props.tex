\documentclass[leqno]{article}
\usepackage{hyperref}
\usepackage{amsmath}
\usepackage[equations]{propositions}
\usepackage{cleveref}

\newtheorem{theorem}{Theorem}

\DeclareNumberedType{P}

\begin{document}

\section{Basic tests}

\subsection{Short items}

\begin{prop}
  \pitem[P] This is proposition P. \label{propP}
  \pitem[Q] This is proposition Q. \label{propQ}
\end{prop}

Reference test: \ref{propP} and \ref{propQ}.

Cleveref test: \cref{propP} and \cref{propQ}.

\subsection{Long items}

\begin{prop}
  \pitem[type=long, Physicalism] Everything is physical. \label{phys}
  \pitem[type=long, Dualism] Mind and body are distinct. \label{dual}
\end{prop}

Reference test: \ref{phys} and \ref{dual}.

\subsection{Numbered items}

\begin{prop}
  \pitem If $p$ then $q$. \label{num1}
  \pitem $p$. \label{num2}
  \pitem Therefore $q$. \label{num3}
\end{prop}

References: \ref{num1}, \ref{num2}, \ref{num3}.

\subsection{Shortcut commands}

\begin{prop}
  \pitem[A] Short item A. \label{shortA}
  \litem[Some Thesis] A long item. \label{longthesis}
  \pitem A numbered item. \label{numshort}
\end{prop}

References: \ref{shortA}, \ref{longthesis}, \ref{numshort}.

\begin{prop}
  \bitem A bullet item. \label{bull1}
\end{prop}

Reference: \ref{bull1}.

\section{Nesting levels}

\begin{prop}
  \pitem First numbered item. \label{nest1}
  \pitem Second numbered item. \label{nest2}
  \begin{prop}
    \pitem Level two sub-item. \label{nest2a}
    \pitem Another level two. \label{nest2b}
    \begin{prop}
      \pitem Level three deep. \label{nest2ai}
      \pitem Another level three. \label{nest2aii}
    \end{prop}
  \end{prop}
  \pitem Third numbered item. \label{nest3}
\end{prop}

References across levels: \ref{nest1}, \ref{nest2a}, \ref{nest2ai}.

\section{Named counters}

\begin{prop}
  \pitem[counter=P] First P-numbered. \label{P1}
  \pitem[counter=P] Second P-numbered. \label{P2}
\end{prop}

References: \ref{P1}, \ref{P2}.

\section{Oref and nref test}

\begin{prop}
  \pitem[P3] The original. \label{myP3}
  \pitem[name={\nref{myP3}$'$}] The modified version. \label{myP3prime}
\end{prop}

Plain ref: \ref{myP3}.

Oref with suffix: \oref[$'$]{myP3}.

Nref: \nref{myP3}.

Ref to modified: \ref{myP3prime}.

\section{References in item names}
\begin{prop}
    \pitem[type=long, Problems with \ref{myP3}]
    Some problem. \label{P3problem}
    \pitem[name={\oref[*]{myP3}}]
    A new version. \label{P3star}
\end{prop}
Reference: \ref{P3problem} and \ref{P3star}.

\section{The label key}

\begin{prop}
  \pitem[label=autolabel, R] Using the label key.
\end{prop}

Reference: \ref{autolabel}.

\section{Shortcut and gloss}

\begin{prop}
  \pitem[name={Long Proposition}, display format={\textit{#1}}, shorthand={LP}, shorthand format={ [#1]}, ref format={(#1)}] A long name with a shorthand. \label{lp}
  \pitem[name={Another Thesis}, display format={\textsc{#1}}, shorthand={AT}] Default shorthand format. \label{at}
  \pitem[name={Glossed Prop}, gloss={roughly, everything is fine}] A proposition with a gloss. \label{gp}
  \pitem[name={Full Example}, display format={\textit{#1}}, shorthand={FE}, shorthand format={ [#1]}, gloss={an example}, gloss format={ --- #1}] Both shorthand and gloss. \label{fe}
\end{prop}

Ref to shorthand item: \ref{lp} (should be (LP)).

Ref to AT: \ref{at} (should be (AT)).

Ref to glossed: \ref{gp} (should be (Glossed Prop) --- gloss does not affect ref).

Ref to full: \ref{fe} (should be (FE) --- shorthand becomes ref).

\section{Propapply and brackets}

\begin{prop}
  \pitem[name={Brack}, ref format={[#1]}] A bracketed item. \label{brk}
  \pitem[type=long, Thesis] A long-format item. \label{thesis}
\end{prop}

Ref to bracketed: \ref{brk} (should be [Brack]).

Oref suffix on brackets: \oref[$'$]{brk} (should be [Brack$'$]).

Nref on brackets: \nref{brk} (should be Brack).

Oref suffix on textsc: \oref[$^*$]{thesis} (should be Thesis$^*$).

Nref on textsc: \nref{thesis} (should be Thesis).

Nref on a ref: \nref{propP} (should be P).

\section{Equation integration}

An equation:
\begin{equation}
  E = mc^2 \label{eq:emc}
\end{equation}

A gather environment:
\begin{gather}
  a + b = c \label{eq:abc} \\
  x^2 + y^2 = z^2 \label{eq:pyth}
\end{gather}

Interleaved with props:
\begin{prop}
  \pitem First premise. \label{eq:prem1}
  \pitem Second premise. \label{eq:prem2}
\end{prop}

A further equation:
\begin{equation}
  F = ma \label{eq:fma}
\end{equation}

Ref to equation: \ref{eq:emc} (should be (\textit{number})).

Ref to gather: \ref{eq:abc} and \ref{eq:pyth}.

Ref to prop: \ref{eq:prem1} and \ref{eq:prem2} (should share counter with equations).

Ref to later equation: \ref{eq:fma}.

Oref suffix on equation: \oref[$'$]{eq:emc} (should be (\textit{number}$'$)).

Nref on equation: \nref{eq:emc} (should be bare number).

\section{Inline prop}

Some text with \begin{inlineprop}\pitem[X] an inline proposition\end{inlineprop} in the middle, and then \begin{inlineprop}\pitem[Y] another one\end{inlineprop} here.

\section{Ptag test}

\subsection{Named short ptag}

\begin{equation}
  E = mc^2 \ptag[P4] \label{eq:ptag1}
\end{equation}

Ref: \ref{eq:ptag1} (should be (P4)).

\subsection{Named long ptag}

\begin{equation}
  F = ma \ptag[type=long, name={Newton's~Law}] \label{eq:newton}
\end{equation}

Ref: \ref{eq:newton} (should be \textsc{Newton's Law}).

\subsection{Ptag with shorthand}

\begin{equation}
  x + y = z \ptag[name={Addition~Principle}, shorthand={AP}] \label{eq:ap}
\end{equation}

Ref: \ref{eq:ap} (should be (AP) since shorthand becomes ref).

\subsection{Ptag in align}

\begin{align}
  a &= b \ptag[P5] \label{eq:ptag2} \\
  c &= d \ptag[P6] \label{eq:ptag3}
\end{align}

Refs: \ref{eq:ptag2} (should be (P5)) and \ref{eq:ptag3} (should be (P6)).

\subsection{Ptag with label key}

\begin{equation}
  g = 9.8 \ptag[label=eq:grav, P7]
\end{equation}

Ref via label key: \ref{eq:grav} (should be (P7)).

\subsection{Oref and nref on ptag}

Oref suffix: \oref[$'$]{eq:ptag1} (should be (P4$'$)).

Nref: \nref{eq:ptag1} (should be P4).

\section{Counter-stepping ptags}

A baseline equation for reference:
\begin{equation}
  a^2 + b^2 = c^2 \label{eq:baseline}
\end{equation}
Ref: \ref{eq:baseline} (should be (14)).

\subsection{Numbered ptag in equation}

\begin{equation}
  v = at \ptag \label{eq:numptag1}
\end{equation}
Ref: \ref{eq:numptag1} (should be (15), consecutive after baseline).

\subsection{Numbered ptag in align}

\begin{align}
  f &= ma \ptag \label{eq:numptag2} \\
  p &= mv \ptag \label{eq:numptag3}
\end{align}
Refs: \ref{eq:numptag2} and \ref{eq:numptag3} (should be (16) and (17), consecutive).

\subsection{Named counter ptag in equation}

\begin{equation}
  s = ut + \tfrac{1}{2}at^2 \ptag[counter=P] \label{eq:Peq}
\end{equation}
Ref: \ref{eq:Peq} (should be (P3)).

Equation counter unchanged: next normal equation should continue from (18).

\subsection{Named counter ptag in align}

\begin{align}
  E_k &= \tfrac{1}{2}mv^2 \ptag[counter=P] \label{eq:Pctr1} \\
  E_p &= mgh \ptag[counter=P] \label{eq:Pctr2}
\end{align}
Refs: \ref{eq:Pctr1} and \ref{eq:Pctr2} (should be (P4) and (P5)).

\subsection{Named counter ptag in gather}

\begin{gather}
  \nabla \cdot \mathbf{E} = \frac{\rho}{\varepsilon_0} \ptag[counter=P] \label{eq:Pgath1} \\
  \nabla \times \mathbf{B} = \mu_0 \mathbf{J} \ptag[counter=P] \label{eq:Pgath2}
\end{gather}
Refs: \ref{eq:Pgath1} and \ref{eq:Pgath2} (should be (P6) and (P7)).

\subsection{Mixed: named counter ptag then normal equation}

\begin{equation}
  G = H - TS \ptag[counter=P] \label{eq:Pmixed}
\end{equation}
Ref: \ref{eq:Pmixed} (should be (P8)).

A normal equation after (should use equation counter, not P counter):
\begin{equation}
  W = Fd \label{eq:afterP}
\end{equation}
Ref: \ref{eq:afterP} (should be (18), equation counter unaffected by P-ptags).

\subsection{Mixed: numbered ptag + normal equation}

\begin{equation}
  \tau = r \times F \ptag \label{eq:numptag4}
\end{equation}
Ref: \ref{eq:numptag4} (should be (19)).

A normal equation after:
\begin{equation}
  P = \frac{W}{t} \label{eq:after}
\end{equation}
Ref: \ref{eq:after} (should be (20), consecutive with ptag above, no gap).

\subsection{Nesting ptags with equations}
\begin{prop}
    \pitem
    Two of Maxwell's equations:
    \begin{inlineprop}
        \begin{align*}
            \nabla \cdot \mathbf{E} = \frac{\rho}{\varepsilon_0} \ptag \label{eq:inner1} \\
            \nabla \times \mathbf{B} = \mu_0 \mathbf{J} \ptag \label{eq:inner2}
        \end{align*}
    \end{inlineprop}
\end{prop}
Ref: \ref{eq:inner1}, \ref{eq:inner2}.

\section{Ref in section heading: \ref{propP}}

This section's title contains a reference to \ref{propP}.

\section{Dimension options}

\subsection{Default spacing (no options)}

\begin{prop}
    \pitem First item with default spacing.
    \pitem Second item with default spacing.
    \pitem Third item with default spacing.
\end{prop}

\subsection{rightmargin=3cm}

\propoptions{rightmargin=3cm}
\begin{prop}
    \pitem First item: the right margin should be noticeably indented.
    \pitem Second item: this text should wrap well before the page edge,
    demonstrating that rightmargin is in effect.
\end{prop}
\propoptions{rightmargin=}

\subsection{leftmargin=4cm}

\propoptions{leftmargin=4cm}
\begin{prop}
    \pitem First item: should be indented far from the left.
    \pitem Second item with large left margin.
\end{prop}
\propoptions{leftmargin=}

\subsection{itemsep=20pt}

\propoptions{itemsep=20pt}
\begin{prop}
    \pitem First item.
    \pitem Second item: there should be a large gap above this.
    \pitem Third item: and above this.
\end{prop}
\propoptions{itemsep=}

\subsection{topsep=30pt}

\propoptions{topsep=30pt}
Text before the list.
\begin{prop}
    \pitem First item: large gap above and below the whole list.
    \pitem Second item.
\end{prop}
Text after the list.
\propoptions{topsep=}

\subsection{tightspacing}

\propoptions{tightspacing}
Text before.
\begin{prop}
    \pitem First item: the list should be compact.
    \pitem Second item: minimal spacing around the list and between items.
    \pitem Third item.
\end{prop}
Text after.

\subsection{nosep}

\propoptions{nosep}
Text before.
\begin{prop}
    \pitem First item: zero spacing everywhere.
    \pitem Second item: no gap between items or around list.
    \pitem Third item.
\end{prop}
Text after.

\section{Equation display format vs.\ ref format}

A normal equation before the group:
\begin{equation}
  F = ma \label{eq:before}
\end{equation}
Ref: \ref{eq:before} (should be (40), parenthesized).

%% Simulate an axiom section: bold display, plain ref, counter reset.
\begingroup
\propoptions{equation display format=\textbf{M#1},
             equation ref format=M#1}
\setcounter{equation}{0}

First axiom:
\begin{equation}
  \forall x\, (x = x) \label{ax:M1}
\end{equation}
Display should be \textbf{M1} (bold, no parens).
Ref: \ref{ax:M1} (should be M1, plain, no parens).

Second axiom:
\begin{equation}
  \forall x\, \forall y\, (x = y \to y = x) \label{ax:M2}
\end{equation}
Display should be \textbf{M2}.
Ref: \ref{ax:M2} (should be M2).
\endgroup

A normal equation after the group (format should revert):
\begin{equation}
  E = mc^2 \label{eq:aftergroup}
\end{equation}
Ref: \ref{eq:aftergroup} (should be (43), parenthesized, counter continues).

Cross-refs from inside the group still work: \ref{ax:M1}, \ref{ax:M2}.

\begin{theorem}
    \begin{enumerate}
        \item 
        Part one of theorem.
        \item
        Part two of theorem
    \end{enumerate}
\end{theorem}

\begin{theorem}
    \begin{prop}
        \ritem 
        Part one of another theorem.
        \ritem
        Part two of that theorem
    \end{prop}
\end{theorem}


\end{document}
